\documentclass[10pt]{article}

\usepackage{hyperref}
\usepackage{xcolor}
\usepackage[sort]{natbib}
\usepackage[framemethod=TikZ]{mdframed}
\usepackage{tikz}
\usetikzlibrary{shapes}
\usetikzlibrary{positioning}

\usepackage{palatino}
\usepackage{fourier-orns}

\begin{document}
\newcommand{\tcstep}{\rightarrow}
\newcommand{\tcmulti}{\tcstep\!\!{}^\ast}


\nolinenumbers

\section*{Short Guide to Papers \hfill{\normalfont\small\it Ben Greenman \quad \today}}

This is a guide to the key papers on migratory typing that I've worked on.
The proposal mentions all of these papers.
The goal here is to briefly communicate the technical contributions in each,
 and to show how they fit together.
Click on the venue of a paper to open a PDF.

\bigskip
\hfill{}\decoone\hfill{}

\paragraph{\href{https://www2.ccs.neu.edu/racket/pubs/popl16-tfgnvf.pdf}{POPL '16}}
Introduces a method to evaluate the performance of a migratory typing system;
 specifically, the method is for programmers who want to add types
 to an arbitrary set of modules in their program.
Evaluates Typed Racket on a suite of benchmark programs.


\paragraph{\href{https://www2.ccs.neu.edu/racket/pubs/gtnffvf-jfp19.pdf}{JFP '19}}
(\emph{written in 2016}\/) \quad
Compares the performance of three implementations of migratory typing and
 introduces a sampling variant of the evaluation method.
Evaluates Typed Racket on an extended suite of benchmark programs;
 the new programs use object-oriented designs.


\paragraph{\href{https://www2.ccs.neu.edu/racket/pubs/pepm18-gm.pdf}{PEPM '18}}
Evaluates Transient Reticulated Python;
 notes major differences from Typed Racket.
On one hand, the worst observed overhead is within one order of magnitude.
On the other hand, the fully-typed version of a program is typically the slowest.
Typed Racket suffers more in the worst case but fares better when fully-typed.


\paragraph{\href{https://www2.ccs.neu.edu/racket/pubs/icfp18-gf.pdf}{ICFP '18}}
Models different approaches to migratory typing as different semantics for a
 common surface language.
Explores the different type soundness guarantees for the semantics and
 presents additional examples.
Compares the performance of a transient Typed Racket (TR-1) to standard
 Typed Racket; this more careful evaluation confirms the general suspicions
 noted in PEPM '19.


\paragraph{\href{https://www2.ccs.neu.edu/racket/pubs/oopsla18-fgsfs.pdf}{OOPSLA '18}}
Uses the POPL '16 method to measure the benefit of an optimization that collapses
 stacks of higher-order contracts to a single coercion.
The collapsing method was implemented by Daniel Feltey and invented by
 Michael Greenberg.
Performance is still poor in many benchmarks after collapsing; the optimization
 is irrelevant in some benchmarks, and needs to be implemented for additional
 higher-order contracts to support others.


\paragraph{\href{https://www2.ccs.neu.edu/racket/pubs/dls18-bg.pdf}{DLS '18}}
Uses the ICFP '18 model to survey developers' preference regarding 
 three different semantics for migratory typing (natural, transient, and erasure).
Respondents preferred a semantics that enforces all types.
The survey did not ask about performance.


\paragraph{\href{https://www2.ccs.neu.edu/racket/pubs/oopsla19-gfd.pdf}{OOPSLA '19}}
Adapts the notion of complete monitoring (from work on higher-order contracts)
 to test whether a semantics enforces all types in a mixed program.
Introduces two properties to measure the quality of blame information in
 languages that fail to satisfy complete monitoring.


\end{document}
