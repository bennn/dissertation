\documentclass[10pt]{article}

\usepackage{xcolor}

\usepackage{hyperref}
\usepackage{alltt}
\usepackage[sort]{natbib}
\usepackage{amsmath}
\usepackage{amssymb}
\usepackage{graphics}
\usepackage[framemethod=TikZ]{mdframed}
\usepackage{subcaption}
\usepackage{pifont}

\usepackage{tikz}
\usetikzlibrary{shapes}
\usetikzlibrary{positioning}

\usepackage{palatino}
\usepackage{fourier-orns}

\begin{document}
\usepackage{listings}
\usepackage{graphicx}
\usepackage{wrapfig}


\lstset{
  escapeinside={(*@}{@*)},
  numbers=left, 
  basicstyle=\small\ttfamily, 
  frame=leftline,
  xleftmargin=.5cm,
  breaklines=true
}

\mathchardef\mhyphen="2D
\overfullrule=1mm

\newcommand{\shorturl}[2]{\href{#1#2}{\texttt{#2}}}

\newcommand{\racketcode}[1]{\texttt{#1}}

\newcommand{\makewordmacro}[2]{\expandafter\newcommand\csname X#1\endcsname[1]{##1#2}}
\makewordmacro{hallow}{ying}
\makewordmacro{eep}{onest}

\newcommand{\tshallow}{\Xhallow{l}}
\newcommand{\tdeep}{\Xeep{h}}
\newcommand{\tShallow}{\Xhallow{L}}
\newcommand{\tDeep}{\Xeep{H}}

\newcommand{\makerefmacro}[1]{\expandafter\newcommand\csname X#1ref\endcsname[2]{##1#1~\ref{##2}}}
\makerefmacro{igure}
\makerefmacro{ection}

\newcommand{\figureref}[1]{\Xigureref{f}{#1}}
\newcommand{\sectionref}[1]{\Xectionref{s}{#1}}

\newcommand{\Figureref}[1]{\Xigureref{F}{#1}}
\newcommand{\Sectionref}[1]{\Xectionref{S}{#1}}

\newcommand{\deliverable}[1]{$#1$-deliverable}
\newcommand{\bm}[1]{\texttt{#1}}

\newcommand{\sfont}[1]{\mathsf{#1}}
\newcommand{\Nsym}{\sfont{N}}
\newcommand{\Tsym}{\sfont{T}}

\newcommand{\boundaryfnfont}[1]{\mathcal{#1}}
\newcommand{\Dsym}{\boundaryfnfont{D}}
\newcommand{\Ssym}{\boundaryfnfont{S}}
\newcommand{\DNsym}{\Dsym_{\Nsym}}
\newcommand{\DTsym}{\Dsym_{\Tsym}}
\newcommand{\SNsym}{\Ssym_{\Nsym}}
\newcommand{\STsym}{\Ssym_{\Tsym}}
\newcommand{\fDN}[2]{\DNsym(#1, #2)}
\newcommand{\fSN}[2]{\SNsym(#1, #2)}
\newcommand{\fDT}[2]{\DTsym(#1, #2)}
\newcommand{\fST}[2]{\STsym(#1, #2)}

\newcommand{\stype}{\tau}
\newcommand{\svar}{x}
\newcommand{\svalue}{v}
\newcommand{\sexpr}{e}
\newcommand{\sint}{i}
\newcommand{\serror}{\sfont{Error}}
\newcommand{\smon}{\sfont{mon}}
\newcommand{\sdyn}{\sfont{dyn}}
\newcommand{\ssta}{\sfont{stat}}

\newcommand{\emon}[2]{\smon\,#1~#2}
\newcommand{\edyn}[2]{\sdyn\,#1~#2}
\newcommand{\esta}[2]{\ssta\,#1~#2}
\newcommand{\epair}[2]{\langle #1, #2 \rangle}
\newcommand{\eapp}[2]{#1~#2}
\newcommand{\efun}[2]{\lambda #1.\, #2}
\newcommand{\estring}[1]{\texttt{``#1''}}

\newcommand{\tarr}[2]{#1\!\Rightarrow\!#2}
\newcommand{\tpair}[2]{#1\!\times\!#2}
\newcommand{\tint}{\sfont{Int}}
\newcommand{\tnat}{\sfont{Nat}}

\newenvironment{mfarray}{\begin{array}{l@{~~}c@{~}l}}{\end{array}}
\newcommand{\sidecond}[1]{\multicolumn{3}{l}{\mbox{\quad #1}}}
\newcommand{\kafka}{\(\mathsf{KafKa}\)}

\nolinenumbers

\section*{Short Guide to Papers \hfill{\normalfont\small\it Ben Greenman \quad \today}}

This is a guide to the key papers on migratory typing that I've worked on.
The proposal mentions all of these papers.
The goal here is to briefly communicate the technical contributions in each,
 and to show how they fit together.
Click on the venue of a paper to open a PDF.

\bigskip
\hfill{}\decoone\hfill{}

\paragraph{\href{https://www2.ccs.neu.edu/racket/pubs/popl16-tfgnvf.pdf}{POPL '16}}
Introduces a method to evaluate the performance of a migratory typing system;
 specifically, the method is for programmers who want to add types
 to an arbitrary set of modules in their program.
Evaluates Typed Racket on a suite of benchmark programs.


\paragraph{\href{https://www2.ccs.neu.edu/racket/pubs/gtnffvf-jfp19.pdf}{JFP '19}}
(\emph{written in 2016, major revision of POPL '16}\/) \quad
Compares the performance of three implementations of migratory typing and
 introduces a sampling variant of the evaluation method.
Evaluates Typed Racket on an extended suite of benchmark programs;
 the new programs use object-oriented designs.


\paragraph{\href{https://www2.ccs.neu.edu/racket/pubs/pepm18-gm.pdf}{PEPM '18}}
Evaluates Transient Reticulated Python;
 notes major differences from Typed Racket.
On one hand, the worst observed overhead is within one order of magnitude.
On the other hand, the fully-typed version of a program is typically the slowest.
Typed Racket suffers more in the worst case but fares better when fully-typed.


\paragraph{\href{https://www2.ccs.neu.edu/racket/pubs/icfp18-gf.pdf}{ICFP '18}}

Presents a systematic, two-part investigation of the design space of migratory typing
 systems.
The first part models different designs as different semantics for a common
 mixed-typed language, compares their type soundness guarantees, and
 present additional examples that are not explained by type soundness.
The second part compares the performance of natural, erasure, and transient
 migratory typing as implementations of the Typed Racket surface language;
 this evaluation confirms the suspicions noted in PEPM '19.


\paragraph{\href{https://www2.ccs.neu.edu/racket/pubs/oopsla19-gfd.pdf}{OOPSLA '19}}

Continues the theoretical investigation of ICFP '18; formally explains the
 differences in behavior that were not accounted for by type soundness using
 the notion of `complete monitoring' from prior work on higher-order contracts.
% A MT system that satisfies CM 


\paragraph{\href{https://www2.ccs.neu.edu/racket/pubs/oopsla18-fgsfs.pdf}{OOPSLA '18}}
Uses the POPL '16 method to measure the benefit of an optimization that collapses
 stacks of higher-order contracts to a single coercion.
The collapsing method was implemented by Daniel Feltey and invented by
 Michael Greenberg.
Performance is still poor in many benchmarks after collapsing; the optimization
 is irrelevant in some benchmarks, and needs to be implemented for additional
 higher-order contracts to support others.


\paragraph{\href{https://www2.ccs.neu.edu/racket/pubs/dls18-bg.pdf}{DLS '18}}
Uses the ICFP '18 model to survey developers' preference regarding 
 three different semantics for migratory typing (natural, transient, and erasure).
Respondents preferred a semantics that enforces all types.
The survey did not ask about performance.


\end{document}
