Three challenges stand between the thesis question and an answer:

\begin{enumerate}
\item Combine \tdeep{} and \tshallow{} migratory typing in a model; formulate and prove safety properties.
\item Implement the model for Racket, re-using the Typed Racket type system.
\item Evaluate the performance of the combined semantics.
\end{enumerate}

\subsection{Challenge 1: Model}

The first challenge is to extend our semantic model of migratory typing to combine
 \tdeep{} and \tshallow{} types in one semantics, rather than as parallel
 developments.
The new model must allow the definition of \tdeep{}-typed code,
 \tshallow{}-typed code, and untyped code in the surface syntax.
All three must be able to share all kinds of values via boundary terms.
In particular, sharing implies that transient code must accept monitored
 values from natural code.
 
The primary goal of the model is to state and prove safety properties for each
 of the three languages in the model.
\tDeep{} types must be trustworthy in any context; they must satisfy a complete
 monitoring theorem and a standard type soundness theorem.
\tShallow{} types must match the type-tag of values.
Untyped code must have well-defined behavior.

A secondary goal of the model is to minimize the amount of run-time checking
 that is needed to ensure safety.
\tShallow{}-typed code may be able to leverage the properties of \tdeep{} types
 to avoid some overhead.
\tDeep{}-typed code may benefit from trusting the constructors of \tshallow{}-typed
 values.
The challenge is to explore the design space, find methods that are likely to
 give a performance benefit, and (time permitting) pursue a full-fledged implementation.

The model must scale to union types, universal types, and recursive types.
That is, these types must either be part of the model or else it must be clear
 how to support them in an implementation --- both for static typing and
 for run-time checks.


\subsection{Challenge 2: Implementation}

The second challenge is to validate the model through an implementation.
Racket is a natural target for such an implementation, because it supports
 \tdeep{} migratory typing and partially supports \tshallow{} migratory typing.
What remains is to extend the partial support for \tshallow{} migratory typing
 and to combine the two styles according to the model.
Time-permitting, I may explore further extensions.


\subsubsection{Primary Goals}

\begin{itemize}
  \item
    Extend the current partial support for \tshallow{} migratory typing
     to accomodate all Typed Racket types that appear in the functional GTP
     benchmarks.
    Each type needs a matching tag-check.
    Some tag-checks are easy to define; for example, the proper tag-check for
     the \racketcode{Symbol} type is the \racketcode{symbol?} predicate.
    Other types present a choice:
     should the check for \racketcode{Listof} ensure a proper list?
     should the check for \racketcode{->*} validate arity and keyword arguments?
    I plan to initially answer ``yes'' to both questions and generally to check
     all possible first-order properties; at least until there is evidence
     that these checks are too expensive.
  \item
    Avoid the Racket contract library.
    Contract combinators have administrative overhead.
    Tag-checks must be realized with simple Racket code wherever possible to
     improve run-time performance and take advantage of compiler optimizations.
  \item
    Interact safely with Typed Racket.
    Statically, \tshallow{} and \tdeep{}-typed code must be able to share type definitions.
    Dynamically, \tdeep{}-typed code must protect itself against \tshallow{} values.
  \item
    Provide relatively fast compilation times for \tshallow{}-Typed racket.
    Some overhead relative to \tdeep{}-Typed Racket may arise from the pass that
    rewrites typed code.
    Anything more than a 10\% slowdown, however, must be studied and explained.
\end{itemize}


\subsubsection{Secondary Goals}

\begin{itemize}
  \item
    Add support for class and object types.
    These types need a tag check, but there are many question about how such
     checks should explore compatible values.
    One idea is to mimic the first-order checks done by the contract system;
     the question is whether those checks suffice for soundness.
  \item
    Investigate a static analysis to remove tag checks.
    Typed Racket employs occurrence typing to propagate information about type-tests.
    A tag-check is a simple type test, and the success of one check has implications
     for the rest of the program.
    For example, if a block of code projects an element of an immutable pair twice in a row,
     then only the first projection should go through a tag check.
  \item
    Adapt the Typed Racket optimizer.
    The current implementation of \tshallow{} types is incompatible with the Typed
     Racket optimizer.
    For one, the implementation outputs code that the optimizer cannot handle.
    More significantly, some optimization passes are inappropriate for \tshallow{}-typed
     code because they rely on \tdeep{} type information.
    \tShallow{}-typed code may still benefit from simple optimizations, however,
     so it is worthwhile to try reusing the optimizer.
    For example, it may specialize an application of \racketcode{+} to expect unboxed numbers.
%  \item
%   Compare performance with and without contracts.
%   If an early implementation of \tshallow{} tag checks relies on the contract library,
%   then it would be useful to measure the performance delta after removing contracts.
%   The data would be a useful reference for other clients of the contract library,
%   and may reveal potential improvements to contracts.
\end{itemize}


\subsection{Challenge 3: Evaluation}

% - functional subset of GTP benchmarks, possibly OO
% - avoid 3^N analysis because ...
%   well its huge and hard, possibly, to understand b/c 2^N for each node
%   but ok, DON'T make this a stated goal

The third challenge is to test the hypothesis that a combination of \tdeep{}
 and \tshallow{} types is better than either one individually.
There are a few ways that a programmer could benefit:
 changing one module from \tdeep{} to \tshallow{} may reduce the cost of type boundaries,
 changing a collection of modules from \tshallow{} to \tdeep{} may remove many tag checks,
 and changing a library from \tdeep{} to \tshallow{} may improve the performance of untyped clients.
% In particular, I posit that changing the Racket math library to use \tshallow{}
% types will improve its performance in untyped programs without ``shifting the problem''
% to \tdeep{}-typed clients.
% I expect all clients to suffer moderate overhead after the change instead
% of the +25x pathologies that programmers currently face.\footnote{\shorturl{https://}{docs.racket-lang.org/math/array.html}}

I plan to use the GTP benchmark programs in a systematic performance evaluation.
It is unclear, however, how to conduct such an evaluation.
An exhaustive study requires $3^N$ measurements for each program with $N$
 modules, which is a large amount of data to collect and interpret.
I propose two ideas for now.

To test the math library, the existing method suffices:
 convert the math library to be \tshallow{}-typed and measure the performance lattice.
This same protocol applies to any benchmarks that depend on typed libraries.

To test the benefits for program authors, who now have three choices for every
module, I propose a path-based metric based on two assumptions.
First, I assume that authors are seeking fully-typed programs.
Second, I assume that the authors are seeking \tdeep{}-typed programs.
For the evaluation, I ask what percentage of paths through the \tdeep{}-typed performance
 lattice have no configurations that exceed a certain overhead, say 10x, given
 the option at each step of converting some modules to \tshallow{} types.
The idea is that a programmer moves up the lattice by adding type annotations
 to one module at a time.
If, at any step, \tdeep{} types lead to unacceptable overhead, the programmer can
 switch to \tshallow{} because both use the same static types.
Once the programmer has added ``enough'' types, switching back to \tdeep{} should
offer a performance boost (in addition to the stronger guarantees of \tdeep{} types).

Both assumptions above threaten the validity of any conclusions drawn from the
evaluation.
Experience with Racket suggests that fully-typed programs are not the norm.
Programmers often end up with partially-typed programs, especially when they
intend to support typed and untyped client programs.
It is also unclear that \tdeep{}-typing offers the best performance for fully-typed
programs.
There are $2^N$ ways of mixing \tdeep{} and \tshallow{} typing in a fully-typed program;
 without a full evaluation, one cannot be sure that a mixed program out-performs
 the \tdeep{} version.
%If I integrate \tshallow{} typing with a variant of the Typed Racket optimizer,
%then I plan to validate this second assumption.

%Possible algorithm for migration:
% if slow, convert the most-recently-changed module.
