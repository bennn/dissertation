\overfullrule=1mm

%% --- override Scribble
\let\ssum\relax
\renewcommand{\SecRefLocal}[3]{\ref{#1}}
\renewcommand{\preDoc}{\bibliographystyle{plainnat}\setcitestyle{numbers}}
%% ... why is this so hard in scribble? just want "no linebreak"
\newcommand{\bgequation}[2]{\begin{equation}#1#2\end{equation}}

\newcommand{\tcstep}{\rightarrow}
\newcommand{\tcmulti}{\tcstep\!\!{}^\ast}
\newcommand{\benchmark}[7]{\vbox{
  {\noindent{#1}~\hrulefill}

  \vspace{1ex}{\small
  \begin{minipage}{0.48\textwidth}\flushleft
    \makebox[1.1cm][l]{\emph{Origin}}   : ~#3\\
    \makebox[1.1cm][l]{\emph{Purpose}}  : ~#4
  \end{minipage}\begin{minipage}{0.48\textwidth}\flushleft
    \makebox[1.15cm][l]{\emph{Author}}   : ~#2\\
    \makebox[1.15cm][l]{\emph{Depends}}  : ~#7
    %#6\hspace{1cm}
  \end{minipage}}

  \vspace{1ex}
  {#5}
  \vspace{2ex}

}}
\newcommand{\inColor}[2]{{\color{#1}{#2}}}
\definecolor{PaleBlue}{rgb}{0.90,0.90,1.0}
\definecolor{LightGray}{rgb}{0.90,0.90,0.90}
\definecolor{CommentColor}{rgb}{0.76,0.45,0.12}
\definecolor{ParenColor}{rgb}{0.52,0.24,0.14}
\definecolor{IdentifierColor}{rgb}{0.15,0.15,0.50}
\definecolor{ResultColor}{rgb}{0.0,0.0,0.69}
\definecolor{ValueColor}{rgb}{0.13,0.55,0.13}
\definecolor{OutputColor}{rgb}{0.59,0.00,0.59}
\newcommand{\RktMeta}[1]{\inColor{IdentifierColor}{#1}}
\newcommand{\twoline}[2]{\parbox[s]{1.8cm}{\flushright\hfill #1\newline#2}}


\mathchardef\mhyphen="2D

\DeclareFontFamily{OT1}{pzc}{}
\DeclareFontShape{OT1}{pzc}{m}{it}{<-> s * [1.1] pzcmi7t}{}
\DeclareMathAlphabet{\mathpzc}{OT1}{pzc}{m}{it}

\mdfdefinestyle{staframestyle}{
  skipabove=0ex,rightmargin=0ex,skipbelow=2ex,leftmargin=2ex, %
  innertopmargin=1.2ex,innerrightmargin=1ex,innerbottommargin=1.2ex,innerleftmargin=1ex, %
  align=center,roundcorner=8pt, %
  outerlinewidth=1pt,middlelinewidth=1pt,innerlinewidth=0.5pt,outerlinecolor=black,middlelinecolor=white,innerlinecolor=black}

\mdfdefinestyle{dynframestyle}{
  skipabove=0ex,rightmargin=0ex,skipbelow=2ex,leftmargin=2ex, %
  innertopmargin=1.2ex,innerrightmargin=1ex,innerbottommargin=1.2ex,innerleftmargin=1ex, %
  align=center,roundcorner=2pt, %
  outerlinewidth=1.3pt,outerlinecolor=black,backgroundcolor=black!07!white}

\newcommand{\todo}[1]{\textbf{TODO}: #1}
\newcommand{\ldotsplus}{\ldots{}_{+}}

\newcommand{\lbl}[2]{\noindent\parbox[t]{\columnwidth}{#1\\#2}}
\newenvironment{rrarray}{\(\begin{array}{l@{\hspace{0.5em}}c@{\hspace{0.5em}}l}}{\end{array}\)}
\newenvironment{displayrrarray}{\smallskip\(\begin{array}{l@{\hspace{0.5em}}c@{\hspace{0.5em}}l}}{\end{array}\)\smallskip}
\newenvironment{langarray}{\(\begin{array}[t]{l@{\hspace{0.5em}}c@{\hspace{0.5em}}l}}{\end{array}\)}
\newcommand{\sidecond}[1]{\multicolumn{3}{l}{\mbox{\quad #1}}}
\newcommand{\wideas}[2]{\makebox[\widthof{$#2$}][l]{$#1$}}

\newtheorem{theorem}{Theorem}[section]
\newtheorem{lemma}[theorem]{Lemma}
\newtheorem{corollary}[theorem]{Corollary}
\newtheorem{example}[theorem]{Example}
\newtheorem{definition}[theorem]{Definition}
\newenvironment{proofsketch}{\renewcommand{\proofname}{Proof Sketch}\begin{proof}}{\end{proof}}

\newcommand{\nsym}{N}
\newcommand{\csym}{C}
\newcommand{\fsym}{F}
\newcommand{\asym}{A}
\newcommand{\tsym}{T}
\newcommand{\esym}{E}
\newcommand{\xsym}{X}
\newcommand{\ysym}{Y}

\newcommand{\nscr}{\mathsf{\nsym}}
\newcommand{\cscr}{\mathsf{\csym}}
\newcommand{\fscr}{\mathsf{\fsym}}
\newcommand{\ascr}{\mathsf{\asym}}
\newcommand{\tscr}{\mathsf{\tsym}}
\newcommand{\escr}{\mathsf{\esym}}
\newcommand{\xscr}{\mathsf{\xsym}}

\newcommand{\nname}{Natural}
\newcommand{\cname}{Co-Natural}
\newcommand{\fname}{Forgetful}
\newcommand{\aname}{Amnesic}
\newcommand{\tname}{Transient}
\newcommand{\ename}{Erasure}

\newcommand{\snredsta}{\vartriangleright}
\newcommand{\snreddyn}{\blacktriangleright}
\newcommand{\scredsta}{\vartriangleright}
\newcommand{\screddyn}{\blacktriangleright}
\newcommand{\scredrule}{\rule[0.5ex]{0.6em}{0.6pt}}
\newcommand{\makearrowhead}[1]{\kern-0.8pt\raisebox{0.16ex}{$#1$}}
\newcommand{\makerred}[1]{\mathrel{\rrarrow_{_{\!#1}}^{\kern-0.1em*}}}
\newcommand{\srredX}{\makerred{}}

\newcommand{\makereductionsymbols}[1]{%
  \expandafter\newcommand\csname nred#1S\endcsname{\snredsta_{_{\!\!\mathsf{\scriptstyle#1}}}}%
  \expandafter\newcommand\csname nred#1D\endcsname{\snreddyn_{_{\!\!\mathsf{\scriptstyle#1}}}}%
  \expandafter\newcommand\csname nred#1Sann\endcsname{\csname nred#1S\endcsname^{\sownerpolicy}}%
  \expandafter\newcommand\csname nred#1Dann\endcsname{\csname nred#1D\endcsname^{\sownerpolicy}}%
  \expandafter\newcommand\csname nred#1Sanns\endcsname{\snredsta_{_{\!\!\thickrrbar{\mathsf{\scriptstyle#1}}}}}%
  \expandafter\newcommand\csname nred#1Danns\endcsname{\snreddyn_{_{\!\!\thickrrbar{\mathsf{\scriptstyle#1}}}}}%
  \expandafter\newcommand\csname cred#1S\endcsname{\mathrel{\scredrule\makearrowhead{\scriptstyle\snredsta}_{_{\!\!\mathsf{\scriptstyle#1}}}}}%
  \expandafter\newcommand\csname cred#1D\endcsname{\mathrel{\scredrule\makearrowhead{\scriptstyle\snreddyn}_{_{\!\!\mathsf{\scriptstyle#1}}}}}%
  \expandafter\newcommand\csname cred#1\endcsname{\mathrel{\rrarrow_{_{\!\mathsf{\scriptstyle#1}}}}}%
  \expandafter\newcommand\csname cred#1anns\endcsname{\mathrel{\longrrarrow_{_{\!\thickrrbar{\mathsf{\scriptstyle#1}}}}}}%
  %\expandafter\newcommand\csname rred#1S\endcsname{\mathrel{\scredrule\makearrowhead{\scriptstyle\snredsta}_{_{\!\!\mathsf{\scriptstyle#1}}}^{\kern-0.1em*}}}%
  %\expandafter\newcommand\csname rred#1D\endcsname{\mathrel{\scredrule\makearrowhead{\scriptstyle\snreddyn}_{_{\!\!\mathsf{\scriptstyle#1}}}^{\kern-0.1em*}}}%
  \expandafter\newcommand\csname rred#1\endcsname{\makerred{\mathsf{\scriptstyle#1}}}%
  \expandafter\newcommand\csname rred#1ann\endcsname{\mathrel{\makebox[0pt][l]{\raisebox{4pt}[2pt]{$\scriptstyle\sownerpolicy$}}\rrarrow_{_{\!\mathsf{\scriptstyle#1}}}^{\kern-0.1em*}}}%
  \expandafter\newcommand\csname rred#1anns\endcsname{\mathrel{\longrrarrow_{_{\!\thickrrbar{\mathsf{\scriptstyle#1}}}}^{\kern-0.1em*}}}%
  \expandafter\newcommand\csname rred#1policy\endcsname[1]{\mathrel{\makebox[0pt][l]{\raisebox{4pt}[2pt]{$\scriptstyle##1$}}\rrarrow_{_{\!\mathsf{\scriptstyle#1}}}^{\kern-0.1em*}}}%
}

\makereductionsymbols{\nsym}
\makereductionsymbols{\csym}
\makereductionsymbols{\fsym}
\makereductionsymbols{\asym}
\makereductionsymbols{\xsym}
\makereductionsymbols{\ysym}
\makereductionsymbols{\tsym}
\makereductionsymbols{\esym}


\newcommand{\nredXsym}{\ooalign{\textcolor[rgb]{0.75,0.75,0.75}{$\blacktriangleright$}\cr\hidewidth\hbox{$\vartriangleright$}\cr}}

\newcommand{\makeunireductionsymbols}[1]{%
  \expandafter\newcommand\csname nred#1Xsym\endcsname{\nredXsym_{_{\!\!\mathsf{\scriptstyle#1}}}}%
  \expandafter\newcommand\csname nred#1X\endcsname{\mathrel{\csname nred#1Xsym\endcsname}}%
  \expandafter\newcommand\csname nred#1Xann\endcsname{\csname nred#1S\endcsname^{\sownerpolicy}}%
  \expandafter\newcommand\csname nred#1Xanns\endcsname{\nredXsym_{_{\!\!\thickrrbar{\mathsf{\scriptstyle#1}}}}}%
}

\makeunireductionsymbols{\tsym}
\makeunireductionsymbols{\esym}

\newcommand{\integers}{\mathbb{Z}}
\newcommand{\naturals}{\mathbb{N}}
\newcommand{\subt}{\mathrel{<:}}
\newcommand{\subteq}{\mathrel{\leqslant\kern-0.3em\raisebox{0.3ex}{:}}}
\newcommand{\floorof}[1]{\lfloor #1 \rfloor}
\newcommand{\tagof}[1]{\floorof{#1}}
\newcommand{\ftagof}[1]{\tagof{#1}}
\newcommand{\cardinality}[1]{|#1|}
\newcommand{\powerset}[1]{\mathpzc{P}(#1)}
\newcommand{\hicup}{\mathrel{\raisebox{0.1ex}{\scalebox{0.9}{\(\cup\)}}}}

\newcommand{\modname}[2]{$\underline{\hbox to #2in{#1}}$\\[-2ex]}
\newcommand{\codett}[1]{\texttt{#1}}
\newcommand{\frombnd}[2]{#1~#2\,\cdot~}
\newcommand{\fromuntyped}[2]{\frombnd{\edynb{#2}{#1}}{\snredsta}}
\newcommand{\fromtyped}[2]{\frombnd{\estab{#2}{#1}}{\snreddyn}}

%% TODO no bullet points, just the >> rule on the left? (tabular, wrap text?)
%%  the left column is too much indented ... do something!
%% TODO add space between logical items ... maybe subitem is right idea
\newcommand{\approachbullet}{--}
\newcommand{\formatapproach}[7]{
  \paragraph{#1} \hfill $\sdyn = $ dynamic to static,\quad$\ssta = $ static to dynamic

  \begin{minipage}[t]{0.48\columnwidth}
  \noindent{}\renewcommand{\labelitemi}{\approachbullet}\begin{itemize}
    \item
      $\fromuntyped{v}{\tint}$
    \item[] #2
    \item
      $\fromuntyped{v}{(\tpair{\tau_0}{\tau_1})}$
    \item[] #3
    \item
      $\fromuntyped{v}{(\tfun{\tau_0}{\tau_1})}$
    \item[] #4
  \end{itemize}
  \end{minipage}\begin{minipage}[t]{0.48\columnwidth}
  \noindent{}\renewcommand{\labelitemi}{\approachbullet}\begin{itemize}
    \item
      $\fromtyped{v}{\tint}$
    \item[] #5
    \item
      $\fromtyped{v}{(\tpair{\tau_0}{\tau_1})}$
    \item[] #6
    \item
      $\fromtyped{v}{(\tfun{\tau_0}{\tau_1})}$
    \item[] #7
  \end{itemize}
  \end{minipage}
  \smallskip
}

\newcommand{\BNFeq}{=}
\newcommand{\eeq}{=}
\newcommand{\feq}{=}
\newcommand{\sassign}{=}
\newcommand{\sdefeq}{\triangleq}
\newcommand{\sabbreveq}{\iff}

\newcommand{\tinst}[2]{#1{\{#2\}}}

\newcommand{\sfont}[1]{\mathsf{#1}}
\newcommand{\typefont}[1]{\mathsf{#1}}
\newcommand{\mffont}[1]{\textit{#1}}
\newcommand{\rrfont}[1]{\mathsf{#1}}
\newcommand{\textstrat}[1]{\textbf{\emph{#1}}}
\newcommand{\wrapfont}[1]{\mathbb{#1}}

\makeatletter
\newcommand{\thickbar}{\mathpalette\@thickbar}
\newcommand{\@thickbar}[2]{{#1\mkern1.5mu\vbox{ % kern = left margin of new character
  \sbox\z@{$#1\mkern-2mu#2\mkern1.4mu$}% kern = left and right bounds of the rule
  \sbox\tw@{$#1\overline{#2}$}%
  \dimen@=\dimexpr\ht\tw@-\ht\z@-1.6\p@\relax % height of rule over base
  \hrule\@height1\p@ % height/thickness of rule
  \vskip\dimen@
  \box\z@}\mkern-1.6mu} % kern = right margin of new character
}
\newcommand{\thickrrbar}{\mathpalette\@thickrrbar}
\newcommand{\@thickrrbar}[2]{{#1\mkern2.0mu\vbox{ % kern = left margin of new character
  \sbox\z@{$#1\mkern0mu#2\mkern0mu$}% kern = ???
  \sbox\tw@{$#1\overline{#2}$}%
  \dimen@=\dimexpr\ht\tw@-\ht\z@-0.6\p@\relax % height of rule over base
  \hrule\@height1\p@ % height/thickness of rule
  \vskip\dimen@
  \box\z@}\mkern-1.5mu} % kern = right margin of new character
}
\newcommand{\thickjbar}{\mathpalette\@thickjbar}
\newcommand{\@thickjbar}[2]{{#1\mkern1.0mu\vbox{ % kern = left margin of new character
  \sbox\z@{$#1\mkern-0.5mu#2\mkern-1.5mu$}% kern = ???
  \sbox\tw@{$#1\overline{#2}$}%
  \dimen@=\dimexpr\ht\tw@-\ht\z@-2.0\p@\relax % height of rule over base
  \hrule\@height0.8\p@ % height/thickness of rule
  \vskip\dimen@
  \box\z@}\mkern0mu} % kern = right margin of new character
}
\makeatother

\newcommand{\substat}[1]{#1_{s}}
\newcommand{\subdyn}[1]{#1_{d}}

\newcommand{\fakebf}[1]{\ooalign{{$#1$}\cr\hidewidth\makebox[4pt][l]{$#1$}\cr}}
\newcommand{\lowersym}[1]{\raisebox{-0.7ex}{$#1$}}
\newcommand{\mktagproj}[1]{\lowersym{\llcorner}\!\!#1\!\!\lowersym{\lrcorner}}

% s = symbol
\newcommand{\stheorem}{\textsc{Theorem}}
\newcommand{\slemma}{\textsc{Lemma}}
\newcommand{\scorollary}{\textsc{Corollary}}
\newcommand{\sexample}{\textsc{Example}}
\newcommand{\sdefinition}{\textsc{Definition}}
\newcommand{\sproposition}{\textsc{Proposition}}
\newcommand{\sconjecture}{\textsc{Conjecture}}
\newcommand{\stopofprogram}{\bullet}
\newcommand{\sunop}{\textit{unop}}
\newcommand{\sbinop}{\textit{binop}}
\newcommand{\ssum}{\sfont{sum}}
\newcommand{\squotient}{\sfont{quotient}}
\newcommand{\sdyn}{\sfont{dyn}}
\newcommand{\sstat}{\sfont{stat}}
\newcommand{\ssta}{\sstat}
\newcommand{\Nstawrapper}{\emph{stat-mon}}
\newcommand{\Ndynwrapper}{\emph{dyn-mon}}
\newcommand{\Astawrapper}{\emph{stat-wrap}}
\newcommand{\Adynwrapper}{\emph{dyn-wrap}}
\newcommand{\sapp}{\sfont{app}}
\newcommand{\slet}{\sfont{let}}
\newcommand{\sinkwd}{\sfont{in}}
\newcommand{\sfst}{\sfont{fst}}
\newcommand{\ssnd}{\sfont{snd}}
\newcommand{\sbndfst}{\sfont{fst}}
\newcommand{\sbndsnd}{\sfont{snd}}
\newcommand{\sbnddom}{\sfont{dom}}
\newcommand{\sbndcod}{\sfont{cod}}
\newcommand{\sincr}{\sfont{add1}}
\newcommand{\sresult}{r}
\newcommand{\sdiverge}{\Omega}
\newcommand{\stagerror}{\sfont{TagErr}}
\newcommand{\tagerrortxt}[1]{\stagerror\,{#1}}
\newcommand{\tagerrorD}{\stagerror}
\newcommand{\redexerror}{\stagerror}
\newcommand{\invarianterror}{\sfont{InvariantErr}}
\newcommand{\tagerrorS}{\invarianterror}
\newcommand{\policyerror}{\sfont{PolicyErr}}
\newcommand{\sboundaryerror}{\sfont{BndryErr}}
\newcommand{\sann}[2]{#1_{\sfont{#2}}}
\newcommand{\sflip}{\mffont{flip}}
\newcommand{\sfresh}{\mffont{fresh}}
\newcommand{\smerge}{\mffont{merge}}
\newcommand{\sshallow}{\mffont{shape-match}}
\newcommand{\sunion}{\cup}
\newcommand{\sdepth}{\mffont{wraps}}
\newcommand{\smondepth}{\mffont{guard-depth}}
\newcommand{\sconcat}{}
\newcommand{\shasbnd}{\mffont{has-boundary}}
\newcommand{\slast}{\mffont{last}}
\newcommand{\sbndeqowners}{\simeq}
\newcommand{\sbndseteqowners}{\mathrel{\supseteq\joinrel\subseteq}}
\newcommand{\smember}{\in}
\newcommand{\snil}{{\cdot}}
\newcommand{\sremove}{\setminus}
\newcommand{\srev}{\mffont{rev}} %{\rule[0.4ex]{0.4em}{0.14ex}}
\newcommand{\snext}{\mffont{next}}
\newcommand{\sintersect}{\cap}
\newcommand{\sholeowner}{\mffont{hole\textsf{-}owner}}
\newcommand{\sholeowneraux}{\mffont{h\textsf{-}own}}
\newcommand{\smonitor}{\mffont{guard}}
\newcommand{\sforget}{\mffont{forget}}
\newcommand{\sD}{\mathpzc{D}}
\newcommand{\sS}{\mathpzc{S}}
\newcommand{\sDH}{\sann{\sD}{H}}
\newcommand{\sSH}{\sann{\sS}{H}}
\newcommand{\sDF}{\sann{\sD}{F}}
\newcommand{\sSF}{\sann{\sS}{F}}
\newcommand{\sDA}{\sann{\sD}{A}}
\newcommand{\sSA}{\sann{\sS}{A}}
\newcommand{\sdynop}{\emph{dynop}}
\newcommand{\sowner}{\ell}
\newcommand{\sowners}{\sowner^{*}}
\newcommand{\sownerset}{\sowners}
\newcommand{\sownerlist}{\thickbar{\sowner}}
\newcommand{\sownerpos}{\sowner_{\mathbf{+}}}
\newcommand{\sownerneg}{\sowner_{\mathbf{-}}}
\newcommand{\sobligation}{k}
\newcommand{\sobligations}{K}
\newcommand{\sguard}{\wrapfont{G}}
\newcommand{\smon}{\sguard}
\newcommand{\sdivzero}{\sfont{DivisionByZero}}
\newcommand{\sblame}{b}
\newcommand{\sbnd}{\sblame}
\newcommand{\sbndcomma}{\raisebox{0.2ex}{$\scriptscriptstyle\blacktriangleleft$}}
\newcommand{\sklist}{\stag^{*}}
\newcommand{\stlist}{\stype^{*}}
\newcommand{\sblist}{\thickbar{\sbnd}}
\newcommand{\sbset}{\sbnd^{*}}
\newcommand{\sblisteq}{=}
\newcommand{\sexpr}{e}
\newcommand{\sexprstat}{\substat{\sexpr}}
\newcommand{\sexprdyn}{\subdyn{\sexpr}}
\newcommand{\ssurface}{s}
\newcommand{\svalue}{v}
\newcommand{\svaluestat}{\substat{\svalue}}
\newcommand{\svaluedyn}{\subdyn{\svalue}}
\newcommand{\sprehist}{\sfont{trace}} % TODO pre-trace?
\newcommand{\shistory}{\wrapfont{T}}
\newcommand{\strace}{\shistory}
\newcommand{\stype}{\tau}
\newcommand{\svar}{x}
\newcommand{\snat}{n}
\newcommand{\sint}{i}
\newcommand{\stypeenv}{\Gamma}
\newcommand{\sownerenv}{\mathpzc{L}}
\newcommand{\sdynproj}{\mathbf{0}}
\newcommand{\sidproj}{\mathbf{1}}
\newcommand{\stagproj}{\mktagproj{\cdot}}
\newcommand{\sXproj}{F}
\newcommand{\sWT}{\vdash}
\newcommand{\sWTO}{\mathrel{\thickjbar{\sWT}}}
\newcommand{\sWTfull}{\sWT_{\sidproj}}
\newcommand{\sWTtag}{\sWT_{\kern-0.3em{\raisebox{0.5ex}{\(\mktagproj{~}\)}}}} % maybe \vdash_K ?
\newcommand{\sWTnone}{\sWT_{\sdynproj}}
\newcommand{\sWTX}{\sWT_{\sXproj}}
\newcommand{\sWS}{\sWT}
\newcommand{\sWD}{\sWT}
\newcommand{\sWL}{\Vdash}
\newcommand{\sWLsingle}{\sWL}
\newcommand{\sWLpath}{\sWL_{p}}
\newcommand{\sWLheap}{\sWL_{h}}
\newcommand{\srexpr}{\lesssim}
\newcommand{\srctx}{\srexpr}
\newcommand{\srvalue}{\srexpr}
\newcommand{\satrexpr}{\eqsim}
\newcommand{\satrctx}{\satrexpr}
\newcommand{\sbehaviorle}{\srexpr}
\newcommand{\sbehaviorge}{\gtrsim}
\newcommand{\sbehavioreq}{\satrexpr}
\newcommand{\sdelta}{\delta}
\newcommand{\sDelta}{\Delta}
\newcommand{\sdeltaH}{\delta_{\hsym}}
\newcommand{\sdeltaA}{\delta_{\asym}}
\newcommand{\saddoblig}{\mffont{add\textsf{-}oblig}}
\newcommand{\saddobligs}{\mffont{add\textsf{-}obligs}}
\newcommand{\saddtrace}{\mffont{add\textsf{-}trace}}
\newcommand{\sgettrace}{\mffont{get\textsf{-}trace}}
\newcommand{\sremtrace}{\mffont{rem\textsf{-}trace}}
\newcommand{\stypefst}{\mffont{fst}}
\newcommand{\stypesnd}{\mffont{snd}}
\newcommand{\stypedom}{\mffont{dom}}
\newcommand{\stypecod}{\mffont{cod}}
\newcommand{\sctx}{E}
\newcommand{\sblistowners}{\mffont{labels}}
\newcommand{\sblistsenders}{\mffont{senders}}
\newcommand{\sbsetsenders}{\sblistsenders}
\newcommand{\svalueowners}{\mffont{owners}}
\newcommand{\stypeobligations}{\sfont{obligations}}
\newcommand{\soutermost}{\sfont{first}}
\newcommand{\spolicy}{\mathbf{p}}
\newcommand{\szeropolicy}{\fakebf{\scriptstyle 0}}
\newcommand{\sownerpolicy}{\fakebf{\scriptstyle\sowner}}
\newcommand{\sownerspolicy}{\fakebf{\scriptstyle\sownerlist}}
\newcommand{\sload}{\sfont{load}}
\newcommand{\sloadH}{\sload_{\hsym}}
\newcommand{\sloadN}{\sloadH}
\newcommand{\sloadT}{\sload_{\tsym}}
\newcommand{\sloadA}{\sload_{\asym}}
\newcommand{\sloadX}{\sload_{\xsym}}
\newcommand{\seval}{\sfont{eval}}
\newcommand{\sevalH}{\seval_{\hsym}}
\newcommand{\sevalHann}{\sevalH^{\sownerpolicy}}
\newcommand{\sevalHanns}{\sevalH^{\sownerspolicy}}
\newcommand{\sevalN}{\sevalH}
\newcommand{\sevalNunl}{\sevalN}
\newcommand{\sevalNann}{\sevalHann}
\newcommand{\sevalNanns}{\sevalHanns}
\newcommand{\sevalT}{\seval_{\tsym}}
\newcommand{\sevalTunl}{\sevalT}
\newcommand{\sevalTann}{\sevalT^{\sownerpolicy}}
\newcommand{\sevalTanns}{\sevalT^{\sownerspolicy}}
\newcommand{\sevalA}{\seval_{\asym}}
\newcommand{\sevalAunl}{\sevalA}
\newcommand{\sevalAann}{\sevalA^{\sownerpolicy}}
\newcommand{\sevalAanns}{\sevalA^{\sownerspolicy}}
\newcommand{\sevalX}{\seval_{\xsym}}
\newcommand{\sevalXunl}{\sevalX}
\newcommand{\sevalXpolicy}[1]{\sevalX^{#1}}
\newcommand{\sevalXann}{\sevalXpolicy{\sownerpolicy}}
\newcommand{\sevalXanns}{\sevalXpolicy{\sownerspolicy}}
\newcommand{\spair}{\times}
\newcommand{\sfun}{\Rightarrow}
\newcommand{\smeet}{\sqcap}
\newcommand{\sjoin}{\sqcup}
\newcommand{\swellformed}{\mathbf{wf}}
\newcommand{\swellformedO}{\thickjbar{\swellformed}}
\newcommand{\ssurfacelang}{Surface Language}
\newcommand{\sevallang}{Evaluation Language}
\newcommand{\sopann}{\mffont{type\textsf{-}ann}}
\newcommand{\scountable}{countable} % 'enumerable' is wider
\newcommand{\sorigins}{Origins}
\newcommand{\sbndhole}{\bullet}
\newcommand{\sclass}{\sfont{class}}
\newcommand{\spublic}{\sfont{public}}
\newcommand{\sconstructor}{\sfont{constructor}}
\newcommand{\sfunkwd}{\sfont{fun}}
\newcommand{\sstatic}{\sfont{static}}
\newcommand{\sdynkwd}{\sfont{dyn}}
\newcommand{\snewkwd}{\sfont{new}}
\newcommand{\sreturn}{\sfont{return}}
\newcommand{\sdefine}{\sfont{define}}
\newcommand{\sdefkwd}{\sfont{def}}
\newcommand{\sfunction}{\sfont{function}}
\newcommand{\sflowint}{\sfont{number}}
\newcommand{\strint}{\sfont{Integer}}
\newcommand{\syntaxho}{Higher-Order Evaluation Syntax}
\newcommand{\syntaxfo}{First-Order Evaluation Syntax}
\newcommand{\syntaxeo}{Erasure Evaluation Syntax}

% t = type
\newcommand{\tdyn}{{\mathpzc U}}
\newcommand{\tnat}{\typefont{Nat}}
\newcommand{\tint}{\typefont{Int}}
\newcommand{\tpair}[2]{#1\!\spair\!#2}
\newcommand{\tfun}[2]{#1\!\sfun\!#2}
\newcommand{\tann}[2]{(#1:#2)}
\newcommand{\tenv}{\stypeenv}
\newcommand{\toptional}{\mbox{\raisebox{0.7ex}{\scalebox{0.9}{\(\stype\)}}\(\!/\!\)\raisebox{-0.4ex}{\scalebox{0.8}{\(\tdyn\)}}}}
\newcommand{\stoptional}{\toptional}

\newcommand{\ownerarrow}{\Uparrow}
\newcommand{\sownerarrow}{\ownerarrow}
\newcommand{\blamearrow}{\uparrow}
\newcommand{\sblamearrow}{\blamearrow}
\newcommand{\forgetowner}{\Downarrow}
\newcommand{\ownertop}{\sowner_{\stopofprogram}}
\newcommand{\sownertop}{\ownertop}
\newcommand{\ownerdyn}{\sowner_{D}}
\newcommand{\ownersta}{\sowner_{S}}
\newcommand{\ownersd}{\sowner_{\tdyn}}
\newcommand{\ownedby}[2]{{#1}^{#2}}

\newcommand{\lobar}{(}
\newcommand{\robar}{)}
% \ulcorner is one idea ... but I like () better

\newcommand{\obars}[2]{\ownedby{\lobar #1 \robar}{#2}}
\newcommand{\obbars}[2]{\ownedby{\lobar\!\lobar #1\robar\!\robar}{#2}}
\newcommand{\ocons}[2]{#1\!\cdot{}\!#2}
\newcommand{\beo}[2]{\sboundaryerror~#1~#2}
\newcommand{\boundaryerror}[2]{\sboundaryerror\,(#1, #2)}
\newcommand{\divisionbyzeroerror}{\sfont{DivErr}}
\newcommand{\obnd}[3]{(#1 \sbndcomma\,#2 \sbndcomma\,#3)}
\newcommand{\oivl}[2]{[#1,\,#2)}

\newcommand{\efun}[2]{\lambda #1.\, #2}
\newcommand{\epair}[2]{\langle #1,#2 \rangle}
\newcommand{\eapp}[3]{\tinst{\sapp}{#1}\,#2~#3}
\newcommand{\eletdecl}[2]{\slet~#1=#2~\sinkwd}
\newcommand{\elet}[3]{\eletdecl{#1}{#2}~#3}
\newcommand{\eunopt}[2]{\tinst{\sunop}{#1}\,#2}
\newcommand{\efst}[2]{\tinst{\sfst}{#1}\,#2}
\newcommand{\esnd}[2]{\tinst{\ssnd}{#1}\,#2}
\newcommand{\eincr}[1]{\sincr\,#1}
\newcommand{\ebinopt}[3]{\tinst{\sbinop}{#1}\,#2\,#3}
\newcommand{\esum}[3]{\tinst{\ssum}{#1}\,#2\,#3}
\newcommand{\equotient}[3]{\tinst{\squotient}{#1}\,#2\,#3}
\newcommand{\estat}[2]{\esta{#1}{#2}}
\newcommand{\eerr}{\mathsf{Err}}
\newcommand{\esubst}[3]{#1[#2\!\leftarrow\!#3]}
\newcommand{\eset}[1]{\{#1\}}
\newcommand{\ehistory}[3]{\shistory^{#1}\,#2\,#3}
\newcommand{\ehopt}[2]{\ehistory{?}{#1}{#2}}
\newcommand{\ehist}[2]{\ehistory{}{#1}{#2}}
\newcommand{\etrace}[2]{\ehist{#1}{#2}}
\newcommand{\eprehist}[2]{\sprehist\,#1\,#2}
\newcommand{\esuffix}[2]{\eprehist{#1}{#2}}
\newcommand{\ebnd}[4]{#1^{#4}\,#2~#3}
\newcommand{\edyn}[2]{\ebnd{\sdyn}{#1}{#2}{}}
\newcommand{\esta}[2]{\ebnd{\ssta}{#1}{#2}{}}
\newcommand{\edynb}[2]{\ebnd{\sdyn}{#1}{#2}{}}
\newcommand{\estab}[2]{\ebnd{\ssta}{#1}{#2}{}}
\newcommand{\emonsuper}[3]{\ebnd{\smon}{#1}{#2}{#3}}
\newcommand{\eguard}[2]{\emonsuper{#1}{#2}{}}
\newcommand{\emon}[2]{\eguard{#1}{#2}}
\newcommand{\emopt}[2]{\emonsuper{#1}{#2}{?}}
\newcommand{\emonplus}[2]{\emonsuper{#1}{#2}{+}}
\newcommand{\emoptplus}[2]{\emonsuper{#1}{#2}{+?}}

\newcommand{\ctxhole}{{[\,]}}
\newcommand{\ctx}{\sctx}
\newcommand{\ctxo}[1]{\sctx^{#1}}
\newcommand{\ctxbf}{Z}
\newcommand{\ctxbars}[2]{[#1]^{#2}}

\newcommand{\fappbase}[3]{#1(#3#2)}
\newcommand{\fappo}[3]{\fappbase{#1}{#2}{#3,}}
\newcommand{\fapp}[2]{\fappbase{#1}{#2}{}}
\newcommand{\ffun}[2]{#1\!\longrightarrow\!#2}
\newcommand{\fbool}{\mathpzc{B}}
\newcommand{\ftrue}{\mathsf{True}}
\newcommand{\ffalse}{\mathsf{False}}
\newcommand{\fflip}[1]{\sflip\,(#1)}
\newcommand{\ffresh}[2]{#1\mbox{ fresh in }#2}
\newcommand{\fmerge}[2]{\smerge\,(#1, #2)}
\newcommand{\funion}[2]{#1 \sunion #2}
\newcommand{\fconcat}[2]{#1 \sconcat #2}
\newcommand{\fdepth}[1]{\sdepth\,(#1)}
\newcommand{\fmondepth}[1]{\smondepth\,(#1)}
\newcommand{\fhasbnd}[2]{\shasbnd\,(#1, #2)}
\newcommand{\fbndfst}[1]{\sbndfst\,(#1)}
\newcommand{\fbndsnd}[1]{\sbndsnd\,(#1)}
\newcommand{\fbnddom}[1]{\sbnddom\,(#1)}
\newcommand{\fbndcod}[1]{\sbndcod\,(#1)}
\newcommand{\fbndeqowners}[2]{#1 \sbndeqowners #2}
\newcommand{\fbndseteqowners}[2]{#1 \sbndseteqowners #2}
\newcommand{\fmember}[2]{#1 \smember #2}
\newcommand{\fintersect}[2]{#1 \sintersect #2}
\newcommand{\flast}[1]{\slast\,(#1)}
\newcommand{\fremove}[2]{#1 \sremove #2}
\newcommand{\fshallow}[2]{\sshallow\,(#1,#2)}
\newcommand{\frev}[1]{\srev\,(#1)} %{{#1}^{\kern-0.3em\srev}}
\newcommand{\fnext}[1]{\snext\,(#1)}
\newcommand{\fholeowner}[1]{\sholeowner\,(#1)}
\newcommand{\fholeowneraux}[2]{\sholeowneraux\,(#1,#2)}
\newcommand{\fmonitor}[1]{\smonitor\,(#1)}
\newcommand{\fforget}[1]{\sforget\,(#1)}
\newcommand{\faddobligs}[2]{\saddobligs\,(#1, #2)}
\newcommand{\faddoblig}[2]{\saddoblig\,(#1, #2)}
\newcommand{\faddtrace}[2]{\saddtrace\,(#1, #2)}
\newcommand{\fprehist}[2]{\faddtrace{#1}{#2}}
\newcommand{\fgettrace}[1]{\sgettrace\,(#1)}
\newcommand{\fremtrace}[1]{\sremtrace\,(#1)}
\newcommand{\ftypefst}[1]{\stypefst\,(#1)}
\newcommand{\ftypesnd}[1]{\stypesnd\,(#1)}
\newcommand{\ftypedom}[1]{\stypedom\,(#1)}
\newcommand{\ftypecod}[1]{\stypecod\,(#1)}
\newcommand{\fcons}[2]{#1,#2}
\newcommand{\feval}[1]{\seval(#1)}
\newcommand{\fevalann}[1]{\seval^{\sownerpolicy}(#1)}
\newcommand{\fevalHunl}[1]{\sevalH(#1)}
\newcommand{\fevalNunl}[1]{\sevalN(#1)}
\newcommand{\fevalHann}[1]{\sevalHann(#1)}
\newcommand{\fevalNann}[1]{\sevalNann(#1)}
\newcommand{\fevalNanns}[1]{\sevalNanns(#1)}
\newcommand{\fevalTunl}[1]{\sevalT(#1)}
\newcommand{\fevalTann}[1]{\sevalTann(#1)}
\newcommand{\fevalTanns}[1]{\sevalTanns(#1)}
\newcommand{\fevalAunl}[1]{\sevalA(#1)}
\newcommand{\fevalAann}[1]{\sevalAann(#1)}
\newcommand{\fevalAanns}[1]{\sevalAanns(#1)}
\newcommand{\fevalXunl}[1]{\sevalX(#1)}
\newcommand{\fevalXann}[1]{\sevalXann(#1)}
\newcommand{\fevalXanns}[1]{\sevalXanns(#1)}
\newcommand{\fevalXpolicy}[2]{\sevalXpolicy{#2}(#1)}
\newcommand{\fXarrow}[2]{#2\!#1}
\newcommand{\fblamearrow}[1]{\fXarrow{#1}{\blamearrow}}
\newcommand{\fownerarrow}[1]{\fXarrow{#1}{\ownerarrow}}
\newcommand{\fload}[1]{\sload(#1)}
\newcommand{\floadH}[1]{\sloadH(#1)}
\newcommand{\floadN}[1]{\sloadN(#1)}
\newcommand{\floadT}[1]{\sloadT(#1)}
\newcommand{\floadA}[1]{\sloadA(#1)}
\newcommand{\floadX}[1]{\sloadX(#1)}
\newcommand{\floadHann}[1]{\sloadH^{\sownerarrow}(#1)}
\newcommand{\floadNann}[1]{\sloadN^{\sownerarrow}(#1)}
\newcommand{\floadTann}[1]{\sloadT^{\sownerarrow}(#1)}
\newcommand{\floadAann}[1]{\sloadA^{\sownerarrow}(#1)}
\newcommand{\floadXann}[1]{\sloadX^{\sownerarrow}(#1)}
\newcommand{\fblistowners}[1]{\sblistowners\,(#1)}
\newcommand{\fblistsenders}[1]{\sblistsenders\,(#1)}
\newcommand{\fbsetowners}[1]{\fblistowners{#1}}
\newcommand{\fbsetsenders}[1]{\fblistsenders{#1}}
\newcommand{\fvalueowners}[1]{\svalueowners\,(#1)}
\newcommand{\ftypeobligations}[1]{\stypeobligations\,(#1)}
\newcommand{\foutermost}[1]{\soutermost\,(#1)}
\newcommand{\fmeet}[1]{\smeet #1}
\newcommand{\fjoin}[1]{\sjoin #1}
\newcommand{\fwellformed}[2]{#1 : #2~\swellformed}
\newcommand{\fopann}[1]{\sopann\,(#1)}
\newcommand{\fdiverge}[2]{#1 #2 \mbox{diverges}}
\newcommand{\fexists}[2]{\exists\,#1.~#2}
\newcommand{\fforall}[2]{\forall\,#1.~#2}
\newcommand{\forigins}[1]{\sorigins{} of the {#1} strategy}
\newcommand{\fproperties}[1]{Properties of #1}
\newcommand{\fdynproj}[1]{\sdynproj(#1)}
\newcommand{\fidproj}[1]{\sidproj(#1)}
\newcommand{\ftagproj}[1]{\mktagproj{#1}}
\newcommand{\fXproj}[1]{\sXproj(#1)}

\newcommand{\rrarrow}{\rightarrow}
\newcommand{\longrrarrow}{\longrightarrow}
\newcommand{\rrsymb}{\vartriangleright}
\newcommand{\rastar}{\rrarrow^{*}}
\newcommand{\rtclosure}[1]{\rastar_{\bigcup\eset{#1}}}
\newcommand{\samplerrsym}{\textbf{r}}
\newcommand{\samplerrarrow}{\mathrel{~{\samplerrsym}~}}
\newcommand{\samplerred}{\makerred{\samplerrsym}}

\newcommand{\propfont}[1]{\mathbf{#1}}
\newcommand{\propts}[1]{\propfont{TS}\fakebf{(}#1\fakebf{)}}
\newcommand{\propcm}{\propfont{CM}}
\newcommand{\propbspath}{\propfont{BS}}
\newcommand{\propbcpath}{\propfont{BC}}
\newcommand{\propbsheap}{\propfont{BS{\mhyphen}h}}
\newcommand{\propbcheap}{\propfont{BC{\mhyphen}h}}
\newcommand{\completemonitoring}{\propcm}

\newcommand{\Xigureref}[2]{{#1}igure~\ref{#2}}
\newcommand{\Xectionref}[2]{{#1}ection~\ref{#2}}
\newcommand{\Xableref}[2]{{#1}able~\ref{#2}}
\newcommand{\Xheoremref}[2]{{#1}heorem~\ref{#2}}
\newcommand{\Xefinitionref}[2]{{#1}efinition~\ref{#2}}
\newcommand{\Xemmaref}[2]{{#1}emma~\ref{#2}}
\newcommand{\Xlauseref}[2]{{#1}lause~\ref{#2}}
\newcommand{\Xawref}[2]{{#1}aw~\ref{#2}}
\newcommand{\Xrogramref}[2]{{#1}rogram~(\ref{#2})}

\newcommand{\figureref}[1]{\Xigureref{f}{#1}}
\newcommand{\sectionref}[1]{\Xectionref{s}{#1}}
\newcommand{\tableref}[1]{\Xableref{t}{#1}}
\newcommand{\figref}[1]{\figureref{#1}}
\newcommand{\secref}[1]{\sectionref{#1}}
\newcommand{\thmref}[1]{\Xheoremref{t}{#1}}
\newcommand{\theoremref}[1]{\thmref{#1}}
\newcommand{\definitionref}[1]{\Xefinitionref{d}{#1}}
\newcommand{\lemmaref}[1]{\Xemmaref{l}{#1}}
\newcommand{\clauseref}[1]{\Xlauseref{c}{#1}}
\newcommand{\lawref}[1]{\Xawref{l}{#1}}
\newcommand{\programref}[1]{\Xrogramref{p}{#1}}

\newcommand{\Figureref}[1]{\Xigureref{F}{#1}}
\newcommand{\Sectionref}[1]{\Xectionref{S}{#1}}
\newcommand{\Tableref}[1]{\Xableref{T}{#1}}
\newcommand{\Secref}[1]{\Sectionref{#1}}
\newcommand{\Figref}[1]{\Figureref{#1}}
\newcommand{\Thmref}[1]{\Xheoremref{T}{#1}}
\newcommand{\Theoremref}[1]{\Thmref{#1}}
\newcommand{\Definitionref}[1]{\Xefinitionref{D}{#1}}
\newcommand{\Lemmaref}[1]{\Xemmaref{L}{#1}}
\newcommand{\Clauseref}[1]{\Xlauseref{C}{#1}}
\newcommand{\Lawref}[1]{\Xawref{L}{#1}}
\newcommand{\Programref}[1]{\Xrogramref{P}{#1}}

\newcommand{\scheck}{\sfont{check}}
\newcommand{\sprevalue}{\sfont{w}}
\newcommand{\sloc}{\sfont{p}}
\newcommand{\sdirection}{\sfont{d}}
\newcommand{\sforwards}{\sfont{\hookrightarrow}}
\newcommand{\sbackwards}{\sfont{\hookleftarrow}}
\newcommand{\stag}{K}
\newcommand{\semptymap}{\emptyset}

\newcommand{\knat}{\tnat}
\newcommand{\kint}{\tint}
\newcommand{\kpair}{\typefont{Pair}}
\newcommand{\kfun}{\typefont{Fun}}

\newcommand{\echeck}[4]{\scheck({#2}, {#3},{#4})}
\newcommand{\echecktwo}[3]{\tinst{\scheck}{#1}\,{#2}\,{#3}}
\newcommand{\echeckone}[2]{\tinst{\scheck}{#1}\,{#2}}
\newcommand{\eloc}{\sloc}
\newcommand{\eloczero}{\bullet} % placeholder variable, for the AT simulation
\newcommand{\locations}{\mathbb{P}}

\newcommand{\conftwo}[2]{#1; #2}
\newcommand{\conf}[3]{#1; #2; #3}
\newcommand{\oconf}[4]{#1; #2; #3; #4}

\newcommand{\vstore}{\mathpzc{H}}
\newcommand{\vrecord}[2]{#1 \mapsto #2}

\newcommand{\bstore}{\mathpzc{B}}
\newcommand{\brecord}[2]{#1 \mapsto #2}
\newcommand{\bcons}[2]{#1 \cup \{#2\}}
\newcommand{\bappend}[2]{#1 \cup #2}

\newcommand{\vstoretype}{\mathpzc{T}}

\newcommand{\Xstore}{\phi}
\newcommand{\nostore}{~~}
\newcommand{\withstores}[1]{#1~\Xstore}
\newcommand{\withstoresprime}[1]{\withstores{#1}'}

\newcommand{\lstore}{\mathpzc{O}}
\newcommand{\lrecord}[2]{#1 \mapsto #2}

\newcommand{\bbstore}{B}
\newcommand{\bbrecord}[2]{#1 \sbndcomma #2}


\newcommand{\sutyped}{U}

\newcommand{\sstoresim}{\sim}
\newcommand{\rstoresim}[2]{#1 \sstoresim #2}

\newcommand{\rbstoresim}[3]{\fbndeqowners{#1}{#3(#2)}}
\newcommand{\rbstoresimbasic}[2]{\fbndeqowners{#1}{#2}}

\newcommand{\fmapref}[2]{#1(#2)}
\newcommand{\smapupdate}{\sfont{update}}
\newcommand{\fmapupdate}[3]{#1[#2 \hicup #3]}
\newcommand{\fmapreplace}[3]{#1[#2 \mapsto #3]}

%%%%%% example extension %%%%%%%
%%%%%%%%%%%%%%%%%%%%%%%%%%%%%%%%

\newcommand{\meapp}[3]{%
  \begin{array}[t]{l@{}l}
    \sapp\{    & {#1}\} \\ 
               & {#2}  \\ 
               & {#3}
  \end{array}}


\newcommand{\mestab}[2]{%
  \begin{array}[t]{l@{\,\,}l}
   \ssta    & {#1} \\ 
            & {#2}  \\ 
  \end{array}}

\newcommand{\medynb}[2]{%
  \begin{array}[t]{l@{\,\,}l}
   \sdyn    & #1 \\ 
               & #2  \\ 
  \end{array}}


\newcommand{\zerowidth}[1]{\makebox[0pt][l]{\(#1\)}}
\newcommand{\zeroheight}[1]{\raisebox{0pt}[0pt][0pt]{\(#1\)}}
\newcommand{\reticmodelsymbol}{{\lambda^{\Downarrow}_{\ell}}}
\newcommand{\shorturl}[2]{\href{#1#2}{\texttt{#2}}}
\newcommand{\missingrules}{(selected rules)}
\newcommand{\bigcheckmark}{\ding{51}}
\newcommand{\bigxmark}{\ding{53}}
\newcommand{\hugecheckmark}{\scalebox{1.25}{\bigcheckmark}}
\newcommand{\hugexmark}{\scalebox{1.3}{\bigxmark}}

\newcommand{\raisedsowner}{\raisebox{0.8ex}{$\scriptstyle\sowner_0$}} % hack for proof terms

\newcommand{\dynhole}{\bullet}

\newcommand{\mtlang}{\dagger}
\newcommand{\dynlang}{\star}
\newcommand{\mtlangann}{{\mtlang}}
\newcommand{\dynlangann}{{\displaystyle\star}}
\newcommand{\kafka}{\(\mathsf{KafKa}\)}
\newcommand{\sqle}{\sqsubseteq}

\newcommand{\deftablemacros}{%
  \newcommand{\tblY}{\mbox{\bigcheckmark}}
  \newcommand{\tblN}{\scalebox{0.8}{\bigxmark}}
  \newcommand{\TSfull}{\sidproj}
  \newcommand{\TStag}{\raisebox{0.5ex}{$\stagproj$}}
  \newcommand{\TSnone}{\sdynproj}
  \newcommand{\Bpath}{\emph{path}}
  \newcommand{\Bheap}{\emph{heap}}
  \newcommand{\hsep}{~~~~~~~~~}
  \newcommand{\offsetbox}[1]{\zerowidth{~~##1}}
  \newcommand{\LE}{\offsetbox{\sbehaviorle}}
  \newcommand{\EQ}{\offsetbox{\sbehavioreq}}}

\newcommand{\csharp}{C\#}
\newcommand{\rrnum}[1]{\mbox{#1 --~~~}}

\newcommand{\grantsponsor}[3]{\href{#3}{#1}}
\newcommand{\grantnum}[2]{{\small\texttt{#2}}}

\newcommand{\definitionsketch}[2]{\paragraph{Definition Sketch \textnormal{(#1)}}\leavevmode\\{\em #2}}
