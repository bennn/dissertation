\documentclass[nonacm,10pt]{acmart}

\usepackage{xcolor}

\usepackage{hyperref}
\usepackage{alltt}
\usepackage[sort]{natbib}
\usepackage{amsmath}
\usepackage{amssymb}
\usepackage{graphics}
\usepackage[framemethod=TikZ]{mdframed}
\usepackage{subcaption}
\usepackage{pifont}

\usepackage{tikz}
\usetikzlibrary{shapes}
\usetikzlibrary{positioning}

\usepackage{listings}
\usepackage{graphicx}
\usepackage{wrapfig}


\lstset{
  escapeinside={(*@}{@*)},
  numbers=left, 
  basicstyle=\small\ttfamily, 
  frame=leftline,
  xleftmargin=.5cm,
  breaklines=true
}

\mathchardef\mhyphen="2D
\overfullrule=1mm

\newcommand{\shorturl}[2]{\href{#1#2}{\texttt{#2}}}

\newcommand{\racketcode}[1]{\texttt{#1}}

\newcommand{\makewordmacro}[2]{\expandafter\newcommand\csname X#1\endcsname[1]{##1#2}}
\makewordmacro{hallow}{ying}
\makewordmacro{eep}{onest}

\newcommand{\tshallow}{\Xhallow{l}}
\newcommand{\tdeep}{\Xeep{h}}
\newcommand{\tShallow}{\Xhallow{L}}
\newcommand{\tDeep}{\Xeep{H}}

\newcommand{\makerefmacro}[1]{\expandafter\newcommand\csname X#1ref\endcsname[2]{##1#1~\ref{##2}}}
\makerefmacro{igure}
\makerefmacro{ection}

\newcommand{\figureref}[1]{\Xigureref{f}{#1}}
\newcommand{\sectionref}[1]{\Xectionref{s}{#1}}

\newcommand{\Figureref}[1]{\Xigureref{F}{#1}}
\newcommand{\Sectionref}[1]{\Xectionref{S}{#1}}

\newcommand{\deliverable}[1]{$#1$-deliverable}
\newcommand{\bm}[1]{\texttt{#1}}

\newcommand{\sfont}[1]{\mathsf{#1}}
\newcommand{\Nsym}{\sfont{N}}
\newcommand{\Tsym}{\sfont{T}}

\newcommand{\boundaryfnfont}[1]{\mathcal{#1}}
\newcommand{\Dsym}{\boundaryfnfont{D}}
\newcommand{\Ssym}{\boundaryfnfont{S}}
\newcommand{\DNsym}{\Dsym_{\Nsym}}
\newcommand{\DTsym}{\Dsym_{\Tsym}}
\newcommand{\SNsym}{\Ssym_{\Nsym}}
\newcommand{\STsym}{\Ssym_{\Tsym}}
\newcommand{\fDN}[2]{\DNsym(#1, #2)}
\newcommand{\fSN}[2]{\SNsym(#1, #2)}
\newcommand{\fDT}[2]{\DTsym(#1, #2)}
\newcommand{\fST}[2]{\STsym(#1, #2)}

\newcommand{\stype}{\tau}
\newcommand{\svar}{x}
\newcommand{\svalue}{v}
\newcommand{\sexpr}{e}
\newcommand{\sint}{i}
\newcommand{\serror}{\sfont{Error}}
\newcommand{\smon}{\sfont{mon}}
\newcommand{\sdyn}{\sfont{dyn}}
\newcommand{\ssta}{\sfont{stat}}

\newcommand{\emon}[2]{\smon\,#1~#2}
\newcommand{\edyn}[2]{\sdyn\,#1~#2}
\newcommand{\esta}[2]{\ssta\,#1~#2}
\newcommand{\epair}[2]{\langle #1, #2 \rangle}
\newcommand{\eapp}[2]{#1~#2}
\newcommand{\efun}[2]{\lambda #1.\, #2}
\newcommand{\estring}[1]{\texttt{``#1''}}

\newcommand{\tarr}[2]{#1\!\Rightarrow\!#2}
\newcommand{\tpair}[2]{#1\!\times\!#2}
\newcommand{\tint}{\sfont{Int}}
\newcommand{\tnat}{\sfont{Nat}}

\newenvironment{mfarray}{\begin{array}{l@{~~}c@{~}l}}{\end{array}}
\newcommand{\sidecond}[1]{\multicolumn{3}{l}{\mbox{\quad #1}}}
\newcommand{\kafka}{\(\mathsf{KafKa}\)}


\title{Model 0: erase boundary info}
\author{Ben Greenman}

\begin{document}
\maketitle


\section{Compilation Model}

This model compiles our three kinds of types down to a common, simple runtime.
The compiler removes data about what languages are sitting around a boundary;
 this is similar to the information that the Racket runtime has about checks
 generated via Typed Racket.

Ingredients:

\begin{itemize}
\item 1 surface language
\item 1 evaluation language + semantics
\item 3 linked surface type judgments
\item 3 linked surface-to-evaluation compilers
\item 3 linked evaluation type judgments
\item 1 consistent ownering judgment
\end{itemize}

The question is: can the language satisfy ``complete monitoring''
 if the compilers remove type and boundary info?

In the end, we need no types and three kinds of cast expression in the
 evaluation language.
Each kind of cast represents the ``most-concerned'' party:
 T and anyone gives T, U to S gives S, and S to U gives U.
The three evaluation sub-languages can mix via the cast expressions and the
 result satisfies both type soundness (exact guarantee varies by surface sub-lang)
 and complete monitoring (for ``honest'' types only).


\subsection{Grammar}

Surface expresssions $\ssurface$,
 types $\stype$,
 type-shapes $\sshape$,
 evaluation expressions $\sexpr$.

\begin{langarray}
  \ssurface & \slangeq &
    \svar \mid \sint \mid \epair{\ssurface}{\ssurface}
    \mid \efun{\svar}{\ssurface}
    \mid \efun{\tann{\svar}{\stype}}{\ssurface}
    \mid \efun{\tann{\svar}{\tfloor{\stype}}}{\ssurface}
    \mid \eunop{\ssurface} \mid \ebinop{\ssurface}{\ssurface} \mid \eapp{\ssurface}{\ssurface}
    \mid \emod{\slang}{\ssurface}
  \\
  \stype & \slangeq &
    \tnat \mid \tint \mid \tpair{\stype}{\stype} \mid \tfun{\stype}{\stype}
  \\
  \sshape & \slangeq &
    \knat \mid \kint \mid \kpair \mid \kfun \mid \kany
  \\
  \stspec & \slangeq &
    \stype \mid \tfloor{\stype} \mid \tdyn
  \\
  \slang & \slangeq &
    \sT \mid \sS \mid \sU
  \\
  \sexpr & \slangeq &
    \svar \mid \sint \mid \epair{\sexpr}{\sexpr}
    \mid \efun{\svar}{\sexpr}
    \mid \efun{\tann{\svar}{\stype}}{\sexpr}
    \mid \efun{\tann{\svar}{\sshape}}{\sexpr}
    \mid \emon{\stype}{\svalue}
    \mid \eunop{\sexpr} \mid \ebinop{\sexpr}{\sexpr} \mid \eapp{\sexpr}{\sexpr}
    \mid \ewrap{\stype}{\sexpr}
    \mid \escan{\sshape}{\sexpr}
    \mid \enoop{\sexpr}
  \\
  \svalue & \slangeq &
    \sint \mid \epair{\svalue}{\svalue}
    \mid \efun{\svar}{\sexpr}
    \mid \efun{\tann{\svar}{\stype}}{\sexpr}
    \mid \efun{\tann{\svar}{\sshape}}{\sexpr}
    \mid \emon{\stype}{\svalue}
  \\
  \serror & \slangeq &
    \stagerror \mid \sscanerror \mid \swraperror \mid \sdivzeroerror
  \\
  \sctx & \slangeq &
    \sctxhole \mid \eunop{\sctx} \mid \ebinop{\sctx}{\sexpr} \mid \ebinop{\svalue}{\sctx}
    \mid \eapp{\sctx}{\sexpr} \mid \eapp{\svalue}{\sctx} \mid \enoop{\sctx}
    \mid \escan{\sshape}{\sctx} \mid \ewrap{\stype}{\sctx}
\end{langarray}


\newpage
\subsection{Surface Typing}

Module expressions separate different surface languages.

A $\tfloor{\stype_0}$ is a (decorated) full type, not a shape.
The typing rules for these decorated types are similar to the rules
 for normal types, but these types give weaker guarantees; the notation
 is meant to illustrate the weaker-ness.

\begin{mathpar}
  \inferrule*{
    \tann{\svar_0}{\tdyn} \in \stypeenv_0
  }{
    \stypeenv_0 \sST \svar_0 : \tdyn
  }

  \inferrule*{
    \tann{\svar_0}{\stype_0} \in \stypeenv_0
  }{
    \stypeenv_0 \sST \svar_0 : \stype_0
  }

  \inferrule*{
    \tann{\svar_0}{\tfloor{\stype_0}} \in \stypeenv_0
  }{
    \stypeenv_0 \sST \svar_0 : \tfloor{\stype_0}
  }

  \inferrule*{
  }{
    \stypeenv_0 \sST \sint_0 : \tdyn
  }

  \inferrule*{
  }{
    \stypeenv_0 \sST \snat_0 : \tnat
  }

  \inferrule*{
  }{
    \stypeenv_0 \sST \snat_0 : \tfloor{\tnat}
  }

  \inferrule*{
  }{
    \stypeenv_0 \sST \sint_0 : \tint
  }

  \inferrule*{
  }{
    \stypeenv_0 \sST \sint_0 : \tfloor{\tint}
  }

  \inferrule*{
    \fcons{\tann{\svar_0}{\tdyn}}{\stypeenv_0} \sST \sexpr_0 : \tdyn
  }{
    \stypeenv_0 \sST \efun{\svar_0}{\sexpr_0} : \tdyn
  }

  \inferrule*{
    \fcons{\tann{\svar_0}{\stype_0}}{\stypeenv_0} \sST \sexpr_0 : \stype_1
  }{
    \stypeenv_0 \sST \efun{\tann{\svar_0}{\stype_0}}{\sexpr_0} : \tfun{\stype_0}{\stype_1}
  }

  \inferrule*{
    \fcons{\tann{\svar_0}{\tfloor{\stype_0}}}{\stypeenv_0} \sST \sexpr_0 : \tfloor{\stype_1}
  }{
    \stypeenv_0 \sST \efun{\tann{\svar_0}{\stype_0}}{\sexpr_0} : \tfloor{\tfun{\stype_0}{\stype_1}}
  }

  \inferrule*{
    \stypeenv_0 \sST \sexpr_0 : \tdyn
  }{
    \stypeenv_0 \sST \emodule{\sulang}{\sexpr_0} : \tdyn
  }

  \inferrule*{
    \stypeenv_0 \sST \sexpr_0 : \stype_0
  }{
    \stypeenv_0 \sST \emodule{\stlang}{\sexpr_0} : \tdyn
  }

  \inferrule*{
    \stypeenv_0 \sST \sexpr_0 : \tfloor{\stype_0}
  }{
    \stypeenv_0 \sST \emodule{\sslang}{\sexpr_0} : \tdyn
  }

  \inferrule*{
    \stypeenv_0 \sST \sexpr_0 : \tdyn
  }{
    \stypeenv_0 \sST \emodule{\sulang}{\sexpr_0} : \stype_0
  }

  \inferrule*{
    \stypeenv_0 \sST \sexpr_0 : \stype_0
  }{
    \stypeenv_0 \sST \emodule{\stlang}{\sexpr_0} : \stype_0
  }

  \inferrule*{
    \stypeenv_0 \sST \sexpr_0 : \tfloor{\stype_0}
  }{
    \stypeenv_0 \sST \emodule{\sslang}{\sexpr_0} : \stype_0
  }

  \inferrule*{
    \stypeenv_0 \sST \sexpr_0 : \tdyn
  }{
    \stypeenv_0 \sST \emodule{\sulang}{\sexpr_0} : \tfloor{\stype_0}
  }

  \inferrule*{
    \stypeenv_0 \sST \sexpr_0 : \stype_0
  }{
    \stypeenv_0 \sST \emodule{\stlang}{\sexpr_0} : \tfloor{\stype_0}
  }

  \inferrule*{
    \stypeenv_0 \sST \sexpr_0 : \tfloor{\stype_0}
  }{
    \stypeenv_0 \sST \emodule{\sslang}{\sexpr_0} : \tfloor{\stype_0}
  }
\end{mathpar}


\newpage
\subsection{Completion (aka Compilation)}

Replace module boundaries with wraps, scans, and no-ops

\begin{mathpar}
  \inferrule*{
  }{
    \stypeenv_0 \sST \svar_0 : \tdyn \scompile \svar_0
  }

  \inferrule*{
  }{
    \stypeenv_0 \sST \svar_0 : \stype_0 \scompile \svar_0
  }

  \inferrule*{
  }{
    \stypeenv_0 \sST \svar_0 : \tfloor{\stype_0} \scompile \svar_0
  }

  \inferrule*{
  }{
    \stypeenv_0 \sST \sint_0 : \tdyn \scompile \sint_0
  }

  \inferrule*{
  }{
    \stypeenv_0 \sST \sint_0 : \stype_0 \scompile \sint_0
  }

  \inferrule*{
  }{
    \stypeenv_0 \sST \sint_0 : \tfloor{\stype_0} \scompile \sint_0
  }

  \inferrule*{
    \stypeenv_0 \sST \sexpr_0 : \tdyn \scompile \sexpr_2
    \\
    \stypeenv_0 \sST \sexpr_1 : \tdyn \scompile \sexpr_3
  }{
    \stypeenv_0 \sST \epair{\sexpr_0}{\sexpr_1} : \tdyn \scompile \epair{\sexpr_2}{\sexpr_3}
  }

  \inferrule*{
    \stypeenv_0 \sST \sexpr_0 : \stype_0 \scompile \sexpr_2
    \\
    \stypeenv_0 \sST \sexpr_1 : \stype_1 \scompile \sexpr_3
  }{
    \stypeenv_0 \sST \epair{\sexpr_0}{\sexpr_1} : \tpair{\stype_0}{\stype_1} \scompile \epair{\sexpr_2}{\sexpr_3}
  }

  \inferrule*{
    \stypeenv_0 \sST \sexpr_0 : \tfloor{\stype_0} \scompile \sexpr_2
    \\
    \stypeenv_0 \sST \sexpr_1 : \tfloor{\stype_1} \scompile \sexpr_3
  }{
    \stypeenv_0 \sST \epair{\sexpr_0}{\sexpr_1} : \tfloor{\tpair{\stype_0}{\stype_1}} \scompile \epair{\sexpr_2}{\sexpr_3}
  }

  \inferrule*{
    \fcons{\tann{\svar_0}{\tdyn}}{\stypeenv_0} \sST \sexpr_0 : \tdyn \scompile \sexpr_1
  }{
    \stypeenv_0 \sST \efun{\svar_0}{\sexpr_0} : \tdyn \scompile \efun{\svar_0}{\sexpr_1}
  }

  \inferrule*{
    \fcons{\tann{\svar_0}{\stype_0}}{\stypeenv_0} \sST \sexpr_0 : \stype_1 \scompile \sexpr_1
  }{
    \stypeenv_0 \sST \efun{\tann{\svar_0}{\stype_0}}{\sexpr_0} : \tfun{\stype_0}{\stype_1} \scompile \efun{\tann{\svar_0}{\stype_0}}{\sexpr_1}
  }

  \inferrule*{
    \fcons{\tann{\svar_0}{\stype_0}}{\stypeenv_0} \sST \sexpr_0 : \stype_1 \scompile \sexpr_1
  }{
    \stypeenv_0 \sST \efun{\tann{\svar_0}{\stype_0}}{\sexpr_0} : \tfun{\stype_0}{\stype_1} \scompile \efun{\tann{\svar_0}{\stype_0}}{\sexpr_1}
  }

  \inferrule*{
    \fcons{\tann{\svar_0}{\tfloor{\stype_0}}}{\stypeenv_0} \sST \sexpr_0 : \tfloor{\stype_1} \scompile \sexpr_1
  }{
    \stypeenv_0 \sST \efun{\tann{\svar_0}{\tfloor{\stype_0}}}{\sexpr_0} : \tfloor{\tfun{\stype_0}{\stype_1}} \scompile \efun{\tann{\svar_0}{\sshape_0}}{\sexpr_1}
  }

  \inferrule*{
    \stypeenv_0 \sST \sexpr_0 : \tdyn \scompile \sexpr_2
    \\
    \stypeenv_0 \sST \sexpr_1 : \tdyn \scompile \sexpr_3
  }{
    \stypeenv_0 \sST \eapp{\sexpr_0}{\sexpr_1} : \tdyn \scompile \eapp{\sexpr_2}{\sexpr_3}
  }

  \inferrule*{
    \stypeenv_0 \sST \sexpr_0 : \tfun{\stype_1}{\stype_0} \scompile \sexpr_2
    \\
    \stypeenv_0 \sST \sexpr_1 : \stype_1 \scompile \sexpr_3
  }{
    \stypeenv_0 \sST \eapp{\sexpr_0}{\sexpr_1} : \stype_0 \scompile \eapp{\sexpr_2}{\sexpr_3}
  }

  \inferrule*{
    \stypeenv_0 \sST \sexpr_0 : \tfloor{\tfun{\stype_1}{\stype_0}} \scompile \sexpr_2
    \\
    \stypeenv_0 \sST \sexpr_1 : \tfloor{\stype_1} \scompile \sexpr_3
  }{
    \stypeenv_0 \sST \eapp{\sexpr_0}{\sexpr_1} : \tfloor{\stype_0} \scompile \escan{\sshape_0}{(\eapp{\sexpr_2}{\sexpr_3})}
  }

  (unop, binop)

  \inferrule*{
    \stypeenv_0 \sST \sexpr_0 : \tdyn \scompile \sexpr_1
  }{
    \stypeenv_0 \sST \emodule{\sulang}{\sexpr_0} : \tdyn \scompile \enoop{\sexpr_1}
  }

  \inferrule*{
    \stypeenv_0 \sST \sexpr_0 : \stype_0 \scompile \sexpr_1
  }{
    \stypeenv_0 \sST \emodule{\stlang}{\sexpr_0} : \tdyn \scompile \ewrap{\stype_0}{\sexpr_1}
  }

  \inferrule*{
    \stypeenv_0 \sST \sexpr_0 : \tfloor{\stype_0} \scompile \sexpr_1
  }{
    \stypeenv_0 \sST \emod{\sslang}{\sexpr_0} : \tdyn \scompile \escan{\sshape_0}{\sexpr_1}
  }

  \inferrule*{
    \stypeenv_0 \sST \sexpr_0 : \tdyn \scompile \sexpr_1
  }{
    \stypeenv_0 \sST \emodule{\sulang}{\sexpr_0} : \stype_0 \scompile \ewrap{\stype_0}{\sexpr_1}
  }

  \inferrule*{
    \stypeenv_0 \sST \sexpr_0 : \stype_0 \scompile \sexpr_1
  }{
    \stypeenv_0 \sST \emodule{\stlang}{\sexpr_0} : \stype_0 \scompile \enoop{\sexpr_1}
  }

  \inferrule*{
    \stypeenv_0 \sST \sexpr_0 : \tfloor{\stype_0} \scompile \sexpr_1
  }{
    \stypeenv_0 \sST \emodule{\sslang}{\sexpr_0} : \stype_0 \scompile \ewrap{\stype_0}{\sexpr_1}
  }

  \inferrule*{
    \stypeenv_0 \sST \sexpr_0 : \tdyn \scompile \sexpr_1
  }{
    \stypeenv_0 \sST \emodule{\sulang}{\sexpr_0} : \tfloor{\stype_0} \scompile \escan{\sshape_0}{\sexpr_1}
  }

  \inferrule*{
    \stypeenv_0 \sST \sexpr_0 : \stype_0 \scompile \sexpr_1
  }{
    \stypeenv_0 \sST \emodule{\stlang}{\sexpr_0} : \tfloor{\stype_0} \scompile \ewrap{\stype_0}{\sexpr_1}
  }

  \inferrule*{
    \stypeenv_0 \sST \sexpr_0 : \tfloor{\stype_0} \scompile \sexpr_1
  }{
    \stypeenv_0 \sST \emodule{\sslang}{\sexpr_0} : \tfloor{\stype_0} \scompile \enoop{\sexpr_1}
  }
\end{mathpar}


\newpage
\subsection{Reduction Relation}
one simple reduction relation for everyone

\begin{rrarray}
  \eunop{\svalue_0} & \snr
  & \stagerror
  \\
  \eunop{\svalue_0} & \snr
  & \sdelta({\sunop}{\svalue_0})
  \\
  \ebinop{\svalue_0}{\svalue_1} & \snr
  & \stagerror
  \\
  \ebinop{\svalue_0}{\svalue_1} & \snr
  & \sdelta(\sbinop, \svalue_0, \svalue_1)
  \\
  \eapp{\svalue_0}{\svalue_1} & \snr &
  \stagerror
  \\
  \eapp{(\efun{\svar_0}{\sexpr_0})}{\svalue_0} & \snr
  & \esubst{\sexpr_0}{\svar_0}{\svalue_0}
  \\
  \eapp{(\efun{\tann{\svar_0}{\stype_0}}{\sexpr_0})}{\svalue_0} & \snr
  & \esubst{\sexpr_0}{\svar_0}{\svalue_0}
  \\
  \eapp{(\efun{\tann{\svar_0}{\sshape_0}}{\sexpr_0})}{\svalue_0} & \snr
  & \sscanerror
  \\
  \eapp{(\efun{\tann{\svar_0}{\sshape_0}}{\sexpr_0})}{\svalue_0} & \snr
  & \esubst{\sexpr_0}{\svar_0}{\svalue_0}
  \\
  \eapp{(\emon{\tfun{\stype_0}{\stype_1}}{\svalue_0})}{\svalue_1} & \snr
  & \ewrap{\stype_1}{(\eapp{\svalue_0}{(\ewrap{\stype_0}{\svalue_1})})}
  \\
  \enoop{\svalue_0} & \snr
  & \svalue_0
  \\
  \escan{\sshape_0}{\svalue_0} & \snr
  & \sscanerror
  \\
  \escan{\sshape_0}{\svalue_0} & \snr
  & \svalue_0
  \\
  \ewrap{\stype_0}{\svalue_0} & \snr
  & \swraperror
  \\
  \ewrap{\tfun{\stype_0}{\stype_1}}{\svalue_0} & \snr
  & \emon{\tfun{\stype_0}{\stype_1}}{\svalue_0}
  \\
  \ewrap{\tpair{\stype_0}{\stype_1}}{\epair{\svalue_0}{\svalue_1}} & \snr
  & \epair{\ewrap{\stype_0}{\svalue_0}}{\ewrap{\stype_1}{\svalue_1}}
  \\
  \ewrap{\sshape_0}{\svalue_0} & \snr
  & \svalue_0
\end{rrarray}


\subsection{Evaluation Typing}

three rules, one for each kind of surface expression

\begin{mathpar}
  \inferrule*{
    \tann{\svar_0}{\sdyn} \in \stypeenv_0
  }{
    \stypeenv_0 \sWTU \svar_0 : \sdyn
  }

  \inferrule*{
  }{
    \stypeenv_0 \sWTU \sint_0 : \sdyn
  }

  \inferrule*{
    \stypeenv_0 \sWTU \sexpr_0 : \sdyn
    \\
    \stypeenv_0 \sWTU \sexpr_1 : \sdyn
  }{
    \stypeenv_0 \sWTU \epair{\sexpr_0}{\sexpr_1} : \sdyn
  }

  \inferrule*{
    \fcons{\tann{\svar_0}{\sdyn}}{\stypeenv_0} \sWTU \sexpr_0 : \sdyn
  }{
    \stypeenv_0 \sWTU \efun{\svar_0}{\sexpr_0} : \sdyn
  }

  \inferrule*{
    \fcons{\tann{\svar_0}{\sshape_0}}{\stypeenv_0} \sWTS \sexpr_0 : \sshape_1
  }{
    \stypeenv_0 \sWTU \efun{\tann{\svar_0}{\sshape_0}}{\sexpr_0} : \sdyn
  }

  \inferrule*{
    \stypeenv_0 \sWTT \svalue_0 : \stype_0
  }{
    \stypeenv_0 \sWTU \emon{\stype_0}{\svalue_0} : \sdyn
  }

  \inferrule*{
    \stypeenv_0 \sWTU \sexpr_0 : \sdyn
  }{
    \stypeenv_0 \sWTU \eunop{\sexpr_0} : \sdyn
  }

  \inferrule*{
    \stypeenv_0 \sWTU \sexpr_0 : \sdyn
    \\
    \stypeenv_0 \sWTU \sexpr_1 : \sdyn
  }{
    \stypeenv_0 \sWTU \ebinop{\sexpr_0}{\sexpr_1} : \sdyn
  }

  \inferrule*{
    \stypeenv_0 \sWTU \sexpr_0 : \sdyn
    \\
    \stypeenv_0 \sWTU \sexpr_1 : \sdyn
  }{
    \stypeenv_0 \sWTU \eapp{\sexpr_0}{\sexpr_1} : \sdyn
  }

  \inferrule*{
    \stypeenv_0 \sWTU \sexpr_0 : \sdyn
  }{
    \stypeenv_0 \sWTU \enoop{\sexpr_0} : \sdyn
  }

  \inferrule*{
    \stypeenv_0 \sWTS \sexpr_0 : \sshape_0
  }{
    \stypeenv_0 \sWTU \enoop{\sexpr_0} : \sdyn
  }

  \inferrule*{
    \stypeenv_0 \sWTU \sexpr_0 : \sdyn
  }{
    \stypeenv_0 \sWTU \escan{\sshape_0}{\sexpr_0} : \sdyn
  }

  \inferrule*{
    \stypeenv_0 \sWTS \sexpr_0 : \sshape_1
  }{
    \stypeenv_0 \sWTU \escan{\sshape_0}{\sexpr_0} : \sdyn
  }

  \inferrule*{
    \stypeenv_0 \sWTT \sexpr_0 : \stype_0
  }{
    \stypeenv_0 \sWTU \ewrap{\stype_0}{\sexpr_0} : \sdyn
  }

  \inferrule*{
  }{
    \stypeenv_0 \sWTU \serror : \sdyn
  }
\end{mathpar}

\begin{mathpar}
  \inferrule*{
    \tann{\svar_0}{\sshape_0} \in \stypeenv_0
  }{
    \stypeenv_0 \sWTS \svar_0 : \sshape_0
  }

  \inferrule*{
  }{
    \stypeenv_0 \sWTS \snat_0 : \tnat
  }

  \inferrule*{
  }{
    \stypeenv_0 \sWTS \sint_0 : \tint
  }

  \inferrule*{
    \stypeenv_0 \sWTS \sexpr_0 : \sshape_0
    \\
    \stypeenv_0 \sWTS \sexpr_1 : \sshape_1
  }{
    \stypeenv_0 \sWTS \epair{\sexpr_0}{\sexpr_1} : \kpair
  }

  \inferrule*{
    \fcons{\tann{\svar_0}{\sdyn}}{\stypeenv_0} \sWTU \sexpr_0 : \sdyn
  }{
    \stypeenv_0 \sWTS \efun{\svar_0}{\sexpr_0} : \kfun
  }

  \inferrule*{
    \fcons{\tann{\svar_0}{\sshape_0}}{\stypeenv_0} \sWTS \sexpr_0 : \sshape_1
  }{
    \stypeenv_0 \sWTS \efun{\tann{\svar_0}{\sshape_0}}{\sexpr_0} : \kfun
  }

  \inferrule*{
    \stypeenv_0 \sWTT \svalue_0 : \stype_0
  }{
    \stypeenv_0 \sWTS \emon{\stype_0}{\svalue_0} : \kfun
  }

  \inferrule*{
    \stypeenv_0 \sWTS \sexpr_0 : \sshape_0
  }{
    \stypeenv_0 \sWTS \eunop{\sexpr_0} : \kany
  }

  \inferrule*{
    \stypeenv_0 \sWTS \sexpr_0 : \sshape_0
    \\
    \stypeenv_0 \sWTS \sexpr_1 : \sshape_1
  }{
    \stypeenv_0 \sWTS \ebinop{\sexpr_0}{\sexpr_1} : \kany
  }

  \inferrule*{
    \stypeenv_0 \sWTS \sexpr_0 : \kfun
    \\
    \stypeenv_0 \sWTS \sexpr_1 : \sshape_0
  }{
    \stypeenv_0 \sWTS \eapp{\sexpr_0}{\sexpr_1} : \kany
  }

  \inferrule*{
    \stypeenv_0 \sWTS \sexpr_0 : \sshape_0
  }{
    \stypeenv_0 \sWTS \enoop{\sexpr_0} : \sshape_0
  }

  \inferrule*{
    \stypeenv_0 \sWTS \sexpr_0 : \sdyn
  }{
    \stypeenv_0 \sWTS \enoop{\sexpr_0} : \kany
  }

  \inferrule*{
    \stypeenv_0 \sWTU \sexpr_0 : \sdyn
  }{
    \stypeenv_0 \sWTS \escan{\sshape_0}{\sexpr_0} : \sshape_0
  }

  \inferrule*{
    \stypeenv_0 \sWTS \sexpr_0 : \sshape_1
  }{
    \stypeenv_0 \sWTS \escan{\sshape_0}{\sexpr_0} : \sshape_0
  }

  \inferrule*{
    \stypeenv_0 \sWTT \sexpr_0 : \stype_0
    \\
    \fshape{\stype_0} = \sshape_0
  }{
    \stypeenv_0 \sWTS \ewrap{\stype_0}{\sexpr_0} : \sshape_0
  }

  \inferrule*{
    \stypeenv_0 \sWTS \sexpr_0 : \sshape_0
    \\
    \fsubt{\sshape_0}{\sshape_1}
  }{
    \stypeenv_0 \sWTS \sexpr_0 : \sshape_1
  }

  \inferrule*{
  }{
    \stypeenv_0 \sWTS \serror : \sdyn
  }
\end{mathpar}

\begin{mathpar}
  \inferrule*{
    \tann{\svar_0}{\stype_0} \in \stypeenv_0
  }{
    \stypeenv_0 \sWTT \svar_0 : \stype_0
  }

  \inferrule*{
  }{
    \stypeenv_0 \sWTT \snat_0 : \tnat
  }

  \inferrule*{
  }{
    \stypeenv_0 \sWTT \sint_0 : \tint
  }

  \inferrule*{
    \stypeenv_0 \sWTT \sexpr_0 : \stype_0
    \\
    \stypeenv_0 \sWTT \sexpr_1 : \stype_1
  }{
    \stypeenv_0 \sWTT \epair{\sexpr_0}{\sexpr_1} : \tpair{\stype_0}{\stype_1}
  }

  \inferrule*{
    \fcons{\tann{\svar_0}{\stype_0}}{\stypeenv_0} \sWTT \sexpr_0 : \stype_1
  }{
    \stypeenv_0 \sWTT \efun{\tann{\svar_0}{\stype_0}}{\sexpr_0} : \tfun{\stype_0}{\stype_1}
  }

  \inferrule*{
    \stypeenv_0 \sWTT \svalue_0 : \sdyn
  }{
    \stypeenv_0 \sWTT \emon{\stype_0}{\svalue_0} : \stype_0
  }

  \inferrule*{
    \stypeenv_0 \sWTT \svalue_0 : \sshape_0
  }{
    \stypeenv_0 \sWTT \emon{\stype_0}{\svalue_0} : \stype_0
  }

  \inferrule*{
    \stypeenv_0 \sWTT \sexpr_0 : \stype_0
    \\
    \sDelta(\sunop, \stype_0) = \stype_1
  }{
    \stypeenv_0 \sWTT \eunop{\sexpr_0} : \stype_1
  }

  \inferrule*{
    \stypeenv_0 \sWTT \sexpr_0 : \stype_0
    \\
    \stypeenv_0 \sWTT \sexpr_1 : \stype_1
    \\
    \sDelta(\sbinop, \stype_0, \stype_1) = \stype_2
  }{
    \stypeenv_0 \sWTT \ebinop{\sexpr_0}{\sexpr_1} : \stype_2
  }

  \inferrule*{
    \stypeenv_0 \sWTT \sexpr_0 : \tfun{\stype_0}{\stype_1}
    \\
    \stypeenv_0 \sWTT \sexpr_1 : \stype_0
  }{
    \stypeenv_0 \sWTT \eapp{\sexpr_0}{\sexpr_1} : \stype_1
  }

  \inferrule*{
    \stypeenv_0 \sWTT \sexpr_0 : \stype_0
  }{
    \stypeenv_0 \sWTT \enoop{\sexpr_0} : \stype_0
  }

  \inferrule*{
    \stypeenv_0 \sWTU \sexpr_0 : \sdyn
  }{
    \stypeenv_0 \sWTT \ewrap{\stype_0}{\sexpr_0} : \stype_0
  }

  \inferrule*{
    \stypeenv_0 \sWTS \sexpr_0 : \sshape_0
  }{
    \stypeenv_0 \sWTT \ewrap{\stype_0}{\sexpr_0} : \stype_0
  }

  \inferrule*{
    \stypeenv_0 \sWTT \sexpr_0 : \stype_0
    \\
    \fsubt{\stype_0}{\stype_1}
  }{
    \stypeenv_0 \sWTT \sexpr_0 : \stype_1
  }

  \inferrule*{
  }{
    \stypeenv_0 \sWTT \serror : \sdyn
  }
\end{mathpar}


\newpage
\subsection{Label Consistency}

all T-owners distinct from S/U (the latter can mix)

\begin{mathpar}
  \inferrule*{
    \tann{\svar_0}{\sowner_0} \in \sownerenv_0
  }{
    \sowner_0; \sownerenv_0 \sWL \svar_0
  }

  \inferrule*{
  }{
    \sowner_0; \sownerenv_0 \sWL \sint_0
  }

  \inferrule*{
    \sowner_0; \sownerenv_0 \sWL \sexpr_0
    \\
    \sowner_0; \sownerenv_0 \sWL \sexpr_1
  }{
    \sowner_0; \sownerenv_0 \sWL \epair{\sexpr_0}{\sexpr_1}
  }

  \inferrule*{
    \sowner_0; \fcons{\tann{\svar_0}{\sowner_0}}{\sownerenv_0} \sWL \sexpr_0
  }{
    \sowner_0; \sownerenv_0 \sWL \efun{\svar_0}{\sexpr_0}
  }

  \inferrule*{
    \sowner_0; \fcons{\tann{\svar_0}{\sowner_0}}{\sownerenv_0} \sWL \sexpr_0
  }{
    \sowner_0; \sownerenv_0 \sWL \efun{\tann{\svar_0}{\sshape_0}}{\sexpr_0}
  }

  \inferrule*{
    \sowner_0; \fcons{\tann{\svar_0}{\sowner_0}}{\sownerenv_0} \sWL \sexpr_0
  }{
    \sowner_0; \sownerenv_0 \sWL \efun{\tann{\svar_0}{\stype_0}}{\sexpr_0}
  }

  \inferrule*{
    \sowner_0; \sownerenv_0 \sWL \sexpr_0
  }{
    \sowner_0; \sownerenv_0 \sWL \eunop{\sexpr_0}
  }

  \inferrule*{
    \sowner_0; \sownerenv_0 \sWL \sexpr_0
    \\
    \sowner_0; \sownerenv_0 \sWL \sexpr_1
  }{
    \sowner_0; \sownerenv_0 \sWL \ebinop{\sexpr_0}{\sexpr_1}
  }

  \inferrule*{
    \sowner_0; \sownerenv_0 \sWL \sexpr_0
    \\
    \sowner_0; \sownerenv_0 \sWL \sexpr_1
  }{
    \sowner_0; \sownerenv_0 \sWL \eapp{\sexpr_0}{\sexpr_1}
  }

  \inferrule*{
    \sowner_1; \sownerenv_0 \sWL \sexpr_0
  }{
    \sowner_0; \sownerenv_0 \sWL \enoop{\obnd{\sowner_0}{}{\sowner_1}}{\sexpr_0}
  }

  \inferrule*{
    \sowner_1; \sownerenv_0 \sWL \sexpr_0
  }{
    \sowner_0; \sownerenv_0 \sWL \escan{\obnd{\sowner_0}{\sshape_0}{\sowner_1}}{\sexpr_0}
  }

  \inferrule*{
    \sowner_1; \sownerenv_0 \sWL \sexpr_0
  }{
    \sowner_0; \sownerenv_0 \sWL \ewrap{\obnd{\sowner_0}{\stype_0}{\sowner_1}}{\sexpr_0}
  }

  \inferrule*{
    \stowner_1; \sownerenv_0 \sWL \sexpr_0
  }{
    \stowner_0; \sownerenv_0 \sWL \ownedby{\sexpr_0}{\stowner_1}
  }

  \inferrule*{
    \ssowner_1; \sownerenv_0 \sWL \sexpr_0
  }{
    \ssowner_0; \sownerenv_0 \sWL \ownedby{\sexpr_0}{\ssowner_1}
  }

  \inferrule*{
    \suowner_0; \sownerenv_0 \sWL \sexpr_0
  }{
    \ssowner_0; \sownerenv_0 \sWL \ownedby{\sexpr_0}{\suowner_0}
  }

  \inferrule*{
    \suowner_1; \sownerenv_0 \sWL \sexpr_0
  }{
    \suowner_0; \sownerenv_0 \sWL \ownedby{\sexpr_0}{\suowner_1}
  }

  \inferrule*{
    \ssowner_0; \sownerenv_0 \sWL \sexpr_0
  }{
    \suowner_0; \sownerenv_0 \sWL \ownedby{\sexpr_0}{\ssowner_0}
  }

\end{mathpar}


\subsection{Properties}

\subsubsection{Notation}

$\ssurface_0 \srr \sexpr_0
 \sdefeq
 \vdash \ssurface_0 : \stspec
  \mbox{ and } \vdash \ssurface_0 : \stspec \scompile \sexpr_1
  \mbox{ and } \sexpr_1 \srr \sexpr_0$


\subsubsection{Theorems}

\begin{theorem}[TS$(\sX,\stypemap)$]
  Language\ $\sX$
  satisfies\ $\fTS{\stypemap}$
  if for all\ $\ssurface_0$
  such that\ $\vdash \ssurface_0 : \stspec$
  holds, one of the following holds:
  \begin{itemize}
    \item $\ssurface_0 \srr \svalue_0$ and\ $\sWTX \svalue_0 : \ftypemap{\stspec}$
    \item $\ssurface_0 \srr \serror$
    \item $\ssurface_0 \srr$ diverges
  \end{itemize}
\end{theorem}

\begin{theorem}[type soundness]\leavevmode
  \begin{itemize}
    \item $\sU$ satisfies\ $\fTS{\stypemapzero}$
    \item $\sS$ satisfies\ $\fTS{\stypemapshape}$
    \item $\sT$ satisfies\ $\fTS{\stypemapone}$
  \end{itemize}
\end{theorem}

\begin{theorem}[complete monitoring]
  If\ $~\vdash \ssurface_0 : \stspec$
  and\ $\sowner_0 \Vdash \ssurface_0$
  and\ $\ssurface_0 \srr \sexpr_0$
  then\ $\sowner_0 \Vdash \sexpr_0$
\end{theorem}


\subsubsection{Lemmas}

\begin{lemma}[$\stypemap$-compile]
  If\ $~\vdash \ssurface_0 : \stspec$
  then\ $\vdash \ssurface_0 : \stspec \scompile \sexpr_0$
  and\ $\sWTX \sexpr_0 : \ftypemap{\stspec}$
\end{lemma}

\begin{lemma}[decomposition]
  For all\ $\sexpr_0$
  there exists unique\ $\sexpr_1, \sctx_0$
  such that\ $\sexpr_0 \sexpreq \finhole{\sctx_0}[\sexpr_1]$
\end{lemma}

\begin{lemma}[type progress]
  If\ $~\vdash \sexpr_0 : \stspec$
  then either\ $\sexpr_0 \in \svalue \cup \serror$
  or\ $\sexpr_0 \scc \sexpr_1$
\end{lemma}

\begin{lemma}[type preservation]
  If\ $~\vdash \sexpr_0 : \stspec$
  and\ $\sexpr_0 \scc \sexpr_1$
  then\ $\vdash \sexpr_1 : \stspec$
\end{lemma}

\begin{lemma}[$\sdelta, \sDelta$ agreement]\leavevmode
  \begin{itemize}
    \item
      If\ $~\sDelta(\sunop, \sdyn) = \sdyn$
      and\ $\vdash \svalue_0 : \sdyn$
      and\ $\fdefined{\sdelta(\sunop, \svalue_0)}$
      then\ $\vdash \sdelta(\sunop, \svalue_0) : \sdyn$
    \item
      If\ $~\sDelta(\sunop, \sshape_0) = \sshape_1$
      and\ $\vdash \svalue_0 : \sshape_0$
      and\ $\fdefined{\sdelta(\sunop, \svalue_0)}$
      then\ $\vdash \sdelta(\sunop, \svalue_0) : \sshape_1$
    \item
      If\ $~\sDelta(\sbinop, \sdyn, \sdyn) = \sdyn$
      and\ $\vdash \svalue_0 : \sdyn$
      and\ $\vdash \svalue_1 : \sdyn$
      and\ $\fdefined{\sdelta(\sbinop, \svalue_0, \svalue_1)}$
      then\ $\vdash \sdelta(\sbinop, \svalue_0, \svalue_1) : \sdyn$
    \item
      If\ $~\sDelta(\sbinop, \sshape_0, \sshape_1) = \sshape_2$
      and\ $\vdash \svalue_0 : \sshape_0$
      and\ $\vdash \svalue_1 : \sshape_1$
      and\ $\fdefined{\sdelta(\sbinop, \svalue_0, \svalue_1)}$
      then\ $\vdash \sdelta(\sbinop, \svalue_0, \svalue_1) : \sshape_2$
  \end{itemize}
\end{lemma}

\begin{lemma}[type substitution]\leavevmode
  \begin{itemize}
    \item
      If\ $~\vdash \efun{\svar_0}{\sexpr_0} : \tdyn$
      and\ $~\vdash \svalue_0 : \tdyn$
      then\ $~\vdash \esubst{\sexpr_0}{\svar_0}{\svalue_0} : \tdyn$
    \item
      If\ $~\vdash \efun{\tann{\svar_0}{\sshape_0}}{\sexpr_0} : \tdyn$
      and\ $~\vdash \svalue_0 : \tdyn$
      and\ $\fshapematch{\sshape_0}{\svalue_0}$
      then\ $~\vdash \esubst{\sexpr_0}{\svar_0}{\svalue_0} : \tdyn$
    \item
      If\ $~\vdash \efun{\svar_0}{\sexpr_0} : \kfun$
      and\ $~\vdash \svalue_0 : \sshape_0$
      then\ $~\vdash \esubst{\sexpr_0}{\svar_0}{\svalue_0} : \sdyn$
    \item
      If\ $~\vdash \efun{\tann{\svar_0}{\sshape_0}}{\sexpr_0} : \kfun$
      and\ $~\vdash \svalue_0 : \sshape_1$
      and\ $\fshapematch{\sshape_0}{\svalue_0}$
      then\ $~\vdash \esubst{\sexpr_0}{\svar_0}{\svalue_0} : \kany$
    \item
      If\ $~\vdash \efun{\tann{\svar_0}{\stype_0}}{\sexpr_0} : \tfun{\stype_0}{\stype_1}$
      and\ $~\vdash \svalue_0 : \stype_0$
      then\ $~\vdash \esubst{\sexpr_0}{\svar_0}{\svalue_0} : \stype_1$
  \end{itemize}
\end{lemma}

\begin{lemma}[S to U]
  If\ $~\vdash \sexpr_0 : \sshape_0$
  then\ $~\vdash \sexpr_0 : \sdyn$
\end{lemma}

\begin{lemma}[type in-hole]
  If\ $~\vdash \finhole{\sctx_0}{\sexpr_0} : \stspec_0$
  then\ $\fexists{\stspec_1} \vdash \sexpr_0 : \stspec_1$
\end{lemma}

\begin{lemma}[type replace]
  If\ $~\vdash \finhole{\sctx_0}{\sexpr_0} : \stspec_0$
  and\ $~\vdash \sexpr_0 : \stspec_1$
  and\ $~\vdash \sexpr_1 : \stspec_1$
  then\ $~\vdash \finhole{\sctx_0}{\sexpr_1} : \stspec_0$
\end{lemma}

\begin{lemma}[boundary]\leavevmode
  \begin{itemize}
    \item
      If\ $~\vdash \svalue_0 : \stspec$
      and\ $\fshapematch{\sshape_0}{\svalue_0}$
      then\ $\vdash \svalue_0 : \sshape_0$
    \item
      If\ $~\vdash \svalue_0 : \sshape_0$
      then\ $\vdash \svalue_0 : \sdyn$
    \item
      If\ $~\vdash \svalue_0 : \stype_0$
      and\ $\ewrap{\stype_0}{\svalue_0} \snr \svalue_1$
      then\ $\vdash \svalue_1 : \ftypemapshape{\stype_0}$
      and\ $\vdash \svalue_1 : \tdyn$
  \end{itemize}
\end{lemma}

\begin{lemma}[owner preservation]
  If\ $~\vdash \sexpr_0 : \stspec$
  and\ $\sowner_0 \Vdash \sexpr_0$
  and\ $\sexpr_0 \snr \sexpr_1$
  then\ $\sowner_0 \Vdash \sexpr_1$
\end{lemma}

\begin{lemma}[label in-hole]
  If\ $\sowner_0 \Vdash \finhole{\sctx_0}{\sexpr_0}$
  then\ $\fexists{\sowner_1} \sowner_1 \Vdash \sexpr_0$
\end{lemma}

\begin{lemma}[label replace]
  If\ $\sowner_0 \Vdash \finhole{\sctx_0}{\sexpr_0}$
  and\ $\sowner_1 \Vdash \sexpr_0$
  and\ $\sowner_1 \Vdash \sexpr_1$
  then\ $\sowner_0 \Vdash \finhole{\sctx_0}{\sexpr_1}$
\end{lemma}


\subsection{Notes}

\begin{itemize}
  \item
    Might be able to prove a ``tag error'' lemma, but the elimination forms
     currently don't tell between static-typed and untyped
  \item
    \ 
\end{itemize}


\end{document}
