\documentclass[nonacm,10pt]{acmart}

\usepackage{hyperref}
\usepackage{xcolor}
\usepackage[sort]{natbib}
\usepackage[framemethod=TikZ]{mdframed}
\usepackage{tikz}
\usetikzlibrary{shapes}
\usetikzlibrary{positioning}

\newcommand{\tcstep}{\rightarrow}
\newcommand{\tcmulti}{\tcstep\!\!{}^\ast}



\title{Model 1: keep boundary info}
\author{Ben Greenman}

\begin{document}
\maketitle

% \section{Notes 2020-02-04}
% 
% Looks bleak.
% 
% Current status (before this document):
% 
% Self-edges:
%   - T => T = noop
%   - S => S = noop
%   - U => U = noop
% 
% T/U:
%   - T => U = wrap
%   - U => T = wrap
% 
% T/S S/U:
%   - T => S = wrap
%   - S => U = noop
%   - U => S = scan
%   - S => T = wrap
% 
% The ideal seems to be weakening the S/T boundary:
% 
%   - T => S = noop
%   - S => T = noop
% 
% Well we can't do exactly that because the S-value might have come from untyped,
% or gone out and in before entering T. And we can't just send T to S because
% S might send it to untyped. We might do this:
% 
%   - T => S = noop
%   - S => U = noop, or wrap if originally-typed value
%   - U => S = scan
%   - S => T = noop, or wrap if originally-typed (or wrong S-typed) value
% 
% But then ....
% 
% - to install the "if" at every S => U boundary, we need wrappers on U values
%   to find the indirection points
% 
% - ditto, to install at every S => T boundary we need wrappers on S values
% 
% - seems best to put those wrappers around U alone .... [[T -- S] -- U] ... but
%   that's not transient anymore, more like forgetful
% 
% So that looks bleak for Transient.
% 
% Anyway, once we have these wrappers, need to dynamically check the
% boundary-lang against the value-lang (and value type maybe). That is going to
% be some work and may not pay off.


%\section{Notes 2020-02-05}
%
%Ok, what if we put wrappers on the T = S boundary?
%
%They cannot be "lite" wrappers. A lite wrapper could maybe skip shape checks,
%but needs to be upgraded for S => U jumps. Impractical.
%
%If we have real wrappers, then we still can't do anything without RTTI.
%
%- T cannot drop wrappers, ever. Thats not CM
%
%- once we have wrappers, need to inspect them to do anything ... can't
%  look at the boundary and check only "wrapped or not"
%
%:/



\section{Model}

TBA

%Ingredients:
%
%\begin{itemize}
%\item 1 surface language
%\item 1 evaluation language + semantics
%\item 3 linked surface type judgments
%\item 3 linked surface-to-evaluation compilers
%\item 3 linked evaluation type judgments
%\item 1 consistent ownering judgment
%\end{itemize}


\subsection{Grammar}

Surface expresssions $\ssurface$,
 types $\stype$,
 type-shapes $\sshape$,
 evaluation expressions $\sexpr$.

\begin{langarray}
  \ssurface & \slangeq &
    \svar \mid \sint \mid \epair{\ssurface}{\ssurface}
    \mid \efun{\svar}{\ssurface}
    \mid \efun{\tann{\svar}{\stype}}{\ssurface}
    \mid \efun{\tann{\svar}{\tfloor{\stype}}}{\ssurface}
    \mid \eunop{\ssurface} \mid \ebinop{\ssurface}{\ssurface} \mid \eapp{\ssurface}{\ssurface}
    \mid \emod{\slang}{\ssurface}
  \\
  \stype & \slangeq &
    \tnat \mid \tint \mid \tpair{\stype}{\stype} \mid \tfun{\stype}{\stype}
  \\
  \sshape & \slangeq &
    \knat \mid \kint \mid \kpair \mid \kfun \mid \kany
  \\
  \stspec & \slangeq &
    \stype \mid \tfloor{\stype} \mid \tdyn
  \\
  \slang & \slangeq &
    \sT \mid \sS \mid \sU
  \\
  \sexpr & \slangeq &
    TBA
    %\svar \mid \sint \mid \epair{\sexpr}{\sexpr}
    %\mid \efun{\svar}{\sexpr}
    %\mid \efun{\tann{\svar}{\stype}}{\sexpr}
    %\mid \efun{\tann{\svar}{\sshape}}{\sexpr}
    %\mid \emon{\stype}{\svalue}
    %\mid \eunop{\sexpr} \mid \ebinop{\sexpr}{\sexpr} \mid \eapp{\sexpr}{\sexpr}
    %\mid \ewrap{\stype}{\sexpr}
    %\mid \escan{\sshape}{\sexpr}
    %\mid \enoop{\sexpr}
  \\
  \svalue & \slangeq &
    \sint \mid \epair{\svalue}{\svalue}
    \mid \efun{\svar}{\sexpr}
    \mid \efun{\tann{\svar}{\stype}}{\sexpr}
    \mid \efun{\tann{\svar}{\sshape}}{\sexpr}
    \mid \emon{\stype}{\svalue}
  \\
  \serror & \slangeq &
    \stagerror \mid \sscanerror \mid \swraperror \mid \sdivzeroerror
  \\
  \sctx & \slangeq &
    \sctxhole \mid \eunop{\sctx} \mid \ebinop{\sctx}{\sexpr} \mid \ebinop{\svalue}{\sctx}
    \mid \eapp{\sctx}{\sexpr} \mid \eapp{\svalue}{\sctx} \mid \enoop{\sctx}
    \mid \escan{\sshape}{\sctx} \mid \ewrap{\stype}{\sctx}
\end{langarray}


\newpage
\subsection{Surface Typing}

Module expressions separate different surface languages.

A $\tfloor{\stype_0}$ is a (decorated) full type, not a shape.
The typing rules for these decorated types are similar to the rules
 for normal types, but these types give weaker guarantees; the notation
 is meant to illustrate the weaker-ness.

\begin{mathpar}
  \inferrule*{
    \tann{\svar_0}{\tdyn} \in \stypeenv_0
  }{
    \stypeenv_0 \sST \svar_0 : \tdyn
  }

  \inferrule*{
    \tann{\svar_0}{\stype_0} \in \stypeenv_0
  }{
    \stypeenv_0 \sST \svar_0 : \stype_0
  }

  \inferrule*{
    \tann{\svar_0}{\tfloor{\stype_0}} \in \stypeenv_0
  }{
    \stypeenv_0 \sST \svar_0 : \tfloor{\stype_0}
  }

  \inferrule*{
  }{
    \stypeenv_0 \sST \sint_0 : \tdyn
  }

  \inferrule*{
  }{
    \stypeenv_0 \sST \snat_0 : \tnat
  }

  \inferrule*{
  }{
    \stypeenv_0 \sST \snat_0 : \tfloor{\tnat}
  }

  \inferrule*{
  }{
    \stypeenv_0 \sST \sint_0 : \tint
  }

  \inferrule*{
  }{
    \stypeenv_0 \sST \sint_0 : \tfloor{\tint}
  }

  \inferrule*{
    \fcons{\tann{\svar_0}{\tdyn}}{\stypeenv_0} \sST \sexpr_0 : \tdyn
  }{
    \stypeenv_0 \sST \efun{\svar_0}{\sexpr_0} : \tdyn
  }

  \inferrule*{
    \fcons{\tann{\svar_0}{\stype_0}}{\stypeenv_0} \sST \sexpr_0 : \stype_1
  }{
    \stypeenv_0 \sST \efun{\tann{\svar_0}{\stype_0}}{\sexpr_0} : \tfun{\stype_0}{\stype_1}
  }

  \inferrule*{
    \fcons{\tann{\svar_0}{\tfloor{\stype_0}}}{\stypeenv_0} \sST \sexpr_0 : \tfloor{\stype_1}
  }{
    \stypeenv_0 \sST \efun{\tann{\svar_0}{\stype_0}}{\sexpr_0} : \tfloor{\tfun{\stype_0}{\stype_1}}
  }

  \inferrule*{
    \stypeenv_0 \sST \sexpr_0 : \tdyn
  }{
    \stypeenv_0 \sST \emodule{\sulang}{\sexpr_0} : \tdyn
  }

  \inferrule*{
    \stypeenv_0 \sST \sexpr_0 : \stype_0
  }{
    \stypeenv_0 \sST \emodule{\stlang}{\sexpr_0} : \tdyn
  }

  \inferrule*{
    \stypeenv_0 \sST \sexpr_0 : \tfloor{\stype_0}
  }{
    \stypeenv_0 \sST \emodule{\sslang}{\sexpr_0} : \tdyn
  }

  \inferrule*{
    \stypeenv_0 \sST \sexpr_0 : \tdyn
  }{
    \stypeenv_0 \sST \emodule{\sulang}{\sexpr_0} : \stype_0
  }

  \inferrule*{
    \stypeenv_0 \sST \sexpr_0 : \stype_0
  }{
    \stypeenv_0 \sST \emodule{\stlang}{\sexpr_0} : \stype_0
  }

  \inferrule*{
    \stypeenv_0 \sST \sexpr_0 : \tfloor{\stype_0}
  }{
    \stypeenv_0 \sST \emodule{\sslang}{\sexpr_0} : \stype_0
  }

  \inferrule*{
    \stypeenv_0 \sST \sexpr_0 : \tdyn
  }{
    \stypeenv_0 \sST \emodule{\sulang}{\sexpr_0} : \tfloor{\stype_0}
  }

  \inferrule*{
    \stypeenv_0 \sST \sexpr_0 : \stype_0
  }{
    \stypeenv_0 \sST \emodule{\stlang}{\sexpr_0} : \tfloor{\stype_0}
  }

  \inferrule*{
    \stypeenv_0 \sST \sexpr_0 : \tfloor{\stype_0}
  }{
    \stypeenv_0 \sST \emodule{\sslang}{\sexpr_0} : \tfloor{\stype_0}
  }
\end{mathpar}


\subsection{Completion (aka Compilation)}

TBA


\subsection{Reduction Relation}

TBA

 
\subsection{Evaluation Typing}

TBA


\subsection{Label Consistency}

TBA


\subsection{Properties}

TBA


\end{document}
